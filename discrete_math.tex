\documentclass{article}
\usepackage[14pt]{extsizes}
\usepackage[T1, T2A]{fontenc}
\usepackage{indentfirst}
\usepackage[russian,english]{babel}
\usepackage{amsmath}
\usepackage{amssymb}
\usepackage[a4paper, margin=2.5cm]{geometry}

\DeclareMathOperator*{\Let}{let}
\DeclareMathOperator*{\If}{if}

\begin{document}

\title{Дискретная математика}
\author{Илья Ковалев}
\date{2024 год}
\maketitle

\section{Информация}

Профессор --- Светлана Олеговна

Почта --- svetlana\_os@mail.ru

Учебники:

\begin{itemize}
	\item Нефедов и Осипова -- Курс Дискретной Математики
	\item Лавров и Максимова -- Сборник задач по теории множеств, мат. логике и теории алгоритмов
\end{itemize}

\section{Теория Множеств}

\subsection{Множества}

Множеству невозможно дать определение без парадоксов.

\subsubsection{Создание множества}

$P(x)$ --- форма от $x$/.

$A = \{x | P(x)\}$

$x^2 - 1 = 0$

$A = \{1, -1\} = \{x | x^2 - 1 = 0\}$

\subsubsection{Равенство}

$\{1, 2\} = \{2, 1, 1, 2, 2, 1\}$

\subsubsection{Подмножества}

$A \subseteq B \Rightarrow \forall x \in A \Rightarrow x \in B$

$A \subseteq B, B \subseteq C \Rightarrow A \subseteq C$

$A = B \Leftrightarrow A \subseteq B, B \subseteq A$

\subsubsection*{Примеры:}

$\mathbb{N} \subseteq \mathbb{Z} \subseteq \mathbb{Q} \subseteq \mathbb{R}$

Пусть $A = \{1, 2\}, B = \{\{1, 2\}, 3\}, C = \{1, 2, 3\}$

$A \not \subseteq B$

\subsubsection{Размер множества}

$m(A) = |A|$ --- количество элементов множества

$\emptyset$ --- пустое множество, $|\emptyset| = 0$

\subsubsection{Множество-степень}

$P(A)$ --- множество-степень --- множество всех подмножеств $A$

Например:

$A = \{1, 2, 3\}$

$P(A) = \{\emptyset, \{1\}, \{2\}\, \{3\}, \{1, 2\}, \{2, 3\}, \{3, 1\}, A\}$

$|P(A)| = 2^{|A|}$

\subsubsection{Абсолютное (универсальное) множество}

Универсальное множество --- $U$ --- множество всех подмножеств

\subsubsection{Операции}

\begin{enumerate}
	\item Объединение. $A \cup B = \{x | x \in A \lor x \in B\}$
	\item Пересечение. $A \cap B = \{x | x \in A \land x \in B\}$
	\item Относительное дополнение.\\
	$A \setminus B = \{x | x \in B \land x \not \in B\}$
	\item Симметрическая разность. $A \oplus B = (A \setminus B) \cup (B \setminus A) = (A \cup B) \setminus (A \cap B)$
	\item Отрицание. $\overline{A} = \{x | x \not \in A\}$
\end{enumerate}

Пример:

Пусть $A \subseteq B$

Тогда:
\begin{itemize}
	\item $A \cap B = A$
	\item $A \cup B = B$
	\item $A \setminus B = \emptyset$
\end{itemize}

\subsubsection*{Примеры:}

Пусть $A = [5; 9), B = [6; 10), C = [4; 8)$

$A \cup B = [5; 10)$

$A \cap C = [5; 8)$

$\overline{B} = (-\infty; 6) \cup [10; +\infty)$

$C \setminus A = \emptyset$

$B \oplus C = [4; 6) \cup [8; 10)$

$\overline{A} \setminus (B \cup \overline{C}) = (4; 9]$

$\overline{(B \setminus A)} \cup \overline{(C \setminus B)} = \mathbb{R}$

\subsubsection{Свойства}

\begin{enumerate}
	\item Коммутативность. $A \cup B = B \cup A$
	\item Ассоциативность. $A \cup (B \cup C) = (A \cup B) \cup C$
	\item Дистрибутивность. $A \cup (B \cap C) = (A \cap B) \cup (A \cap C)$
	\item Идемпотентность. $A \cup A = A$
	\item Закон Де-Моргана. $\overline{A \cup B} = \overline{A} \cap \overline{B}$
	\item Поглощение. $A \cup (A \cap B) = A$
	\item Расщепление. $A = (A \cup B) \cap (A \cup \overline{B})$
	\item $\overline{\overline{A}} = A$
\end{enumerate}

\subsubsection{Функция $\psi$}

$A \subseteq U, A \rightarrow \psi(x): U \rightarrow \{0, 1\}$

$\psi_A(x) = \begin{cases}
	1, \text{если} x \in A\\
	0, \text{если} x \not \in A
\end{cases}$

\begin{enumerate}
	\item $\psi_{A \cap B} = \psi_A\psi_B$
	\item $\psi_{A \cup B} = \psi_A + \psi_B - \psi_A\psi_B$
	\item $\psi_{A \oplus B} = \psi_A + \psi_B - 2\psi_A\psi_B$\\
	Заметим, что $\psi_A^2 = \psi_A$, тогда $\psi_{A \oplus B} = (\psi_A - \psi_B)^2 = |\psi_A - \psi_B|$
	\item $\psi_{\overline{A}} = 1 - \psi_A$
	\item $\psi_{A \setminus B} = \psi_A - \psi_A\psi_B$
\end{enumerate}

\subsubsection{Табличный метод}

\begin{tabular}{c|c||c||c||c||c||c}
	$\psi_A$ & $\psi_B$ & $\psi_{A \cap B}$ &
	$\psi_{A \cup B}$ & $\psi_{\overline{A}}$ &
	$\psi_{A \setminus B}$ & $\psi_{A \oplus B}$ \\
	\hline
	1 & 1 & 1 & 1 & 0 & 0 & 0 \\
	1 & 0 & 0 & 1 & 0 & 1 & 1 \\
	0 & 1 & 0 & 1 & 1 & 0 & 1\\
	0 & 0 & 0 & 0 & 1 & 0 & 0
\end{tabular}

Пример:

$A \setminus (B \setminus C) = (A \setminus B) \cup (A \cap C)$

\begin{tabular}{c|c|c|c|c||c|c|c}
	$\psi_A$ & $\psi_B$ & $\psi_C$ & $\psi_{B \setminus C}$ &
	$\psi_{A \setminus (B \setminus C)}$ &
	$\psi_{A \setminus B}$ & $\psi_{A \cap C}$ & $\psi_\cup$
	\\
	\hline
	0 & 0 & 0 & 0 & 0 & 0 & 0 & 0 \\
	0 & 0 & 1 & 1 & 0 & 0 & 0 & 0 \\
	0 & 1 & 0 & 0 & 0 & 0 & 0 & 0 \\
	0 & 1 & 1 & 0 & 0 & 0 & 0 & 0 \\
	\hline
	1 & 0 & 0 & 0 & 1 & 1 & 0 & 1 \\
	1 & 0 & 1 & 1 & 0 & 1 & 1 & 1 \\
	1 & 1 & 0 & 0 & 1 & 0 & 0 & 0 \\
	1 & 1 & 1 & 0 & 1 & 0 & 1 & 1 \\
\end{tabular}

\subsubsection{Доказательство утверждением}

Условия:
\begin{enumerate}
	\item $A \subseteq B$
	\item $A \cap B = A$
	\item $A \cup B = A$
\end{enumerate}
эквивалентны.

{\Large \underline{Доказательство $1 \Rightarrow 2$:}}

Дано: $A \subseteq B$

Доказать: $A \cap B = A$

\begin{enumerate}
	\item $A \cap B \subseteq A$\\
	Пусть $x \in A \cap B \Rightarrow x \in A$
	\item $A \subseteq A \cap B$\\
	Пусть $x \in A \stackrel{\text{по усл}}{\Rightarrow} x \in B
	\Rightarrow x \in A \cap B$
\end{enumerate}

{\Large \underline{Доказательство $2 \Rightarrow 3$:}}

Дано: $A \cap B = A$

Доказать: $A \cup B = B$

$A \cup B = (A \cap B) \cup B = B$

{\Large \underline{Доказательство $3 \Rightarrow 1$:}}

Дано: $A \cup B = B$

Доказать: $A \subseteq B$

От противного:

Предположим $A \not \subseteq B \Rightarrow \exists x \in A \land x \not \in B$

Так как $x \in A \Rightarrow x \in A \cap B
\stackrel{\text{по усл}}{\Rightarrow} x \in B$ --- противоречие

Следовательно, $A \subseteq B$

\subsubsection{ДЗ}

Лаврова \& Максимов: Часть I, $\P$ 1, №11, 12, 14

\section{Бинарные отношения}

$\rho \subseteq A \times B$

$A = B = X$

$\langle x, y \rangle \in \rho \eqcirc x \rho y$

\subsection{Свойства}

\textbf{Бинарное отношение} $\rho$, заданное на множестве $X$, называется:
\begin{enumerate}
	\item \textbf{Рефлексивным} если $\forall x \in X, x \rho x$\\
	Примеры: $x = x$, $x \parallel x$
	\item \textbf{Симметричным} если $\forall x, y \in X, x \rho y \Rightarrow y \rho x$\\
	Примеры: $x = x$, $x \perp x$
	\item \textbf{Антисимметричным} если $\forall x, y \in X, x \rho y \land y \rho x \Rightarrow x = y$\\
	Примеры: $x \ge y$, $x \le y$, $x > y$, $X \subseteq Y$
	\item \textbf{Транзитивным} если $\forall x, y, z \in X, x \rho y \land y \rho z \Rightarrow x \rho z$\\
	Примеры: $x = y \land y = z$, $x \parallel y \land y \parallel z$
	\item \textbf{Ассимметричным} если $\forall x, y \in X, x \rho y \Rightarrow \overline{y \rho x}$
\end{enumerate}

\subsection{Специальные бинарные отношения}

\subsubsection{Отношение эквивалентности}

\begin{enumerate}
	\item Рефлексивно
	\item Симметрично
	\item Транзитивно
\end{enumerate}

Пример:

\begin{itemize}
	\item $x = x$
	\item $x = y \Rightarrow y = x$
	\item $x = y \land y = z \Rightarrow x = z$
\end{itemize}

\subsubsection{Отношение (частичного) порядка}

\textbf{Частично упорядоченное} множество --- множество, на котором задано отношение частичного порядка.

\begin{enumerate}
	\item Рефлексивно
	\item Антисимметрично
	\item Транзитивно
\end{enumerate}

Пример:

\begin{itemize}
	\item $x \le x$
	\item $x \le y \land y \le x \Rightarrow x = y$
	\item $x \le y \land y \le z \Rightarrow x \le z$
\end{itemize}

{\Large Утверждение:}

Пусть $\rho$ --- эквивалентнсть на $X$

Тогда $\rho$:

\begin{enumerate}
	\item Задает на $X$ разбиение на классы эквивалентности.
	\item Если на $X$ задано разбиение, то отношение $\rho$, заданное так, что $x \rho y \Leftrightarrow x$ и $y \in$ одном подмножестве
\end{enumerate}

\subsection{Разбиение}

$A = \bigcup \limits_{i = 1}^n A_i$, так что:

\begin{enumerate}
	\item $\forall i, j \in [1; n], i \ne j, A_i \cap A_j = \emptyset$
	\item $\forall i \in [1; n], A_i \ne \emptyset$
\end{enumerate}

Разбиение $A$ --- $P(A) = \{A_1, A_2 \dots A_n\}$

\textbf{Наибольший элемент}: элемент $a \in X$ --- наибольший, если \\
$\forall y \in X : a \le y$

\textbf{Максимальный элемент}: элемент $a \in X$ --- максимальный, если \\
$\not \exists y \in X : a \lessapprox y$

\textbf{Наименьший элемент}: элемент $a \in X$ --- наименьший, если \\
$\forall y \in X : a \ge y$

\textbf{Минимальный элемент}: элемент $a \in X$ --- минимальный, если \\
$\not \exists y \in X : a \gtrapprox y$

Два частично упорядоченных множества $X$ и $Y$ называются \\
\textbf{изоморфными}, если существует биекция

\subsubsection*{№13(б)}

\begin{gather*}
	A \subseteq B \cup C \Leftrightarrow A \cap \overline{B} \subseteq C
\end{gather*}

{\large Дано:}
\[ A \subseteq B \cup C \]

{\large Доказать:}
\[ A \cap \overline{B} \subseteq C \]

{\large Доказательство:}
\begin{gather*}
	\Let x \in A \cap \overline{B} \Rightarrow x \in A \land x \not \in B \\
	x \in a \stackrel{\text{усл}}{\Rightarrow} x \in B \cup C \Rightarrow x \in B \lor x \in C \\
	\begin{cases}
		x \in B \lor x \in C \\
		x \not \in B
	\end{cases} \Rightarrow x \in C
\end{gather*}

{\large Дано:}
\[ A \cap \overline{B} \subseteq C \]

{\large Доказать:}
\[ A \subseteq B \cup C \]

{\large Доказательство:}
\begin{gather*}
	\Let x \in A \\
	\If x \in B \Rightarrow x \in B \cup C \\
	\If x \not \in B \Rightarrow x \in A \cap \overline{B} \stackrel{\text{усл}}{\Rightarrow} x \in C \Rightarrow x \in B \cup C
\end{gather*}

\subsection{Метод доказательства критериев}

\begin{gather*}
	A \subseteq B \cup C \Leftrightarrow A \cap \overline{B} \subseteq C \\
	A \setminus (B \cup C) = \emptyset \Leftrightarrow (A \cap \overline{B}) \setminus C = \emptyset \\
	\psi_{A \setminus (B \cup C)} = 0 \Leftrightarrow \psi_{(A \cap \overline{B}) \setminus C} = 0 \\
	\psi_{A \setminus (B \cup C)} = \psi_{(A \cap \overline{B}) \setminus C}
\end{gather*}	

\subsection{Виды бинарных отношений}

\subsubsection{Функция}

\begin{gather*}
	f \subseteq X \times Y \\
	\forall x \in X, \forall y_1 \in Y, \forall y_2 \in Y, \\
	\langle x, y_1 \rangle \in f \land \langle x, y_2 \rangle \in f \Rightarrow y_1 = y_2
\end{gather*}

\begin{gather*}
	\langle x, y \rangle \in f \subseteq A \times B \\
	x f y \\
	y = f(x) \\
	f: A \to B \\
	A \mapsto B
\end{gather*}

\subsubsection{Сюръекция}
\begin{gather*}
	f : X \to Y \\
	\forall y \in Y \exists x \in X : y = f(x)
\end{gather*}

\subsubsection{Инъекция}
\begin{gather*}
	f : X \to Y \\
	\forall x_1, x_2 \in X, x_1 = x_2 \Leftrightarrow f(x_1) = f(x_2)
\end{gather*}

\subsubsection{Биекция}
\begin{gather*}
	f : X \leftrightarrow Y \\
	\forall y \in Y \exists x \in X : y = f(x) \land x = f(y)
\end{gather*}

\subsection{Функции над множествами}
\begin{gather*}
	f: X \to Y \\
	A \subseteq X \\
	f(A) = \{y\ |\ y = f(x), x \in A\}
\end{gather*}

\subsubsection{Свойства}

\begin{enumerate}
	\item $f(A \cup B) = f(A) \cup f(B)$
	\item $f(A \cap B) \subseteq f(A) \cap f(B)$
	\item $f(A \setminus B) \supseteq f(A) \setminus f(B)$
	\item $f^{-1}(A \cup B) = f^{-1}(A) \cup f^{-1}(B)$
	\item $f^{-1}(A \cap B) = f^{-1}(A) \cap f^{-1}(B)$
	\item $f^{-1}(A \setminus B) = f^{-1}(A) \setminus f^{-1}(B)$
\end{enumerate}

\subsection{Мощность множества \\
(Кардинальное число множества)}

\[
f(x): A \to B \land f \text{ --- биекция} \Rightarrow A \sim B
\]

$A$ и $B$ равномощны (эквивалентны).

Свойства эквивалентности:

\begin{enumerate}
	\item Рефлексивность: $A \sim A$
	\item Симметричность: $A \sim B \Rightarrow B \sim A$
	\item Транзитивность: $A \sim B \land B \sim C \Rightarrow A \sim C$
\end{enumerate}

$|A| = [A]_{\sim}$ --- мощность множества $A$

$\overline{\overline{A}} = |A| = \text{card} A$

\[
	N_n = \{1, 2, \dots, n\} \land A = \{a_1, a_2, \dots, a_n \} \Rightarrow A \sim N_n
\]

Множество, не являющееся конечным, является \textbf{бесконечным}.

\subsection{Прямое (Декартово) произведение}
\begin{gather*}
	X, Y \\
	X \times Y = \{ \langle x, y \rangle | x \in X, y \in Y \} \\
	X \times Y \ne Y \times X \\
	X \times X = X^2 \\
	D_{X \times Y} = X \text{ --- область определения} \\
	E_{X \times Y} = Y \text{ --- область значений}
\end{gather*}

\subsubsection*{Примеры:}

Доказать:
\[
L = (A \times B) \cap (C \times D) = (A \cap C) \times (B \cap D) = R
\]

Доказательство:
\begin{enumerate}
	\item \begin{gather*}
		L \subseteq R \\
		\Let \langle x, y \rangle \in L \Rightarrow
		\begin{cases}
			\begin{cases}
				x \in A \\
				y \in B
			\end{cases} \\
			\begin{cases}
				x \in C \\
				y \in D
			\end{cases}
		\end{cases} \Rightarrow
		\begin{cases}
			x \in A \cap C \\
			y \in B \cap D
		\end{cases} \Rightarrow
		\langle x, y \rangle \in R
	\end{gather*}
	\item $R \subseteq L$ аналогично
\end{enumerate}

\subsubsection{Операции}

\begin{enumerate}
	\item $\cup, \cap, \dots$
	\item Обратное отношение
	\[
		\rho^{-1} = \{ \langle x, y \rangle | x, y \in X : \langle y, x \rangle \in \rho \}
	\]
	\item Композиция
	\[
		\rho_1 \circ \rho_2 = \{ \langle x, y \rangle | \exists z \in X :
		\langle x, z \rangle \in \rho_2, \langle z, y \rangle \in \rho_1 \}
	\]
	То есть
	\begin{gather*}
		f \circ g = f(g(x)) \\
		\begin{cases}
			\langle x, z \rangle: z = g(x) \\
			\langle z, y \rangle: y = f(z)
		\end{cases} \Rightarrow \langle x, y \rangle: y = f(g(x))
	\end{gather*}
\end{enumerate}

Свойства:

\begin{enumerate}
	\item \[ (\rho^{-1})^{-1} = \rho \]
	\item \[ (\rho_1 \circ \rho_2)^{-1} = \rho_2^{-1} \circ \rho_1^{-1} \]
\end{enumerate}

\subsection{Условия свойств}

\[
d = \{ \langle x, x \rangle | \forall x \in X \}
\]
$d$ --- отношение-диагональ

\subsubsection{Рефлексивности}
\[ d \subseteq \rho \]

\subsubsection{Симметричности}
\[ \rho = \rho^{-1} \]

\subsubsection{Антисимметричности}
\[ \rho \cap \rho^{-1} \subseteq d \]

\subsubsection{Асимметричности}
\[ \rho \cap \rho^{-1} = \emptyset \]
\subsubsection{Частичный порядок}

$\rho$ на $X$ --- ЧП, если $\rho$ --- рефлексивно, антисимметрично, транзитивно


\subsubsection{Транзитивность}
\[ \rho^2 \subseteq \rho \]

\subsection{Специальные бинарные отношения}

\subsubsection{Частичный порядок}

$\rho$ на $X$ --- ЧП, если $\rho$ --- рефлексивно, антисимметрично, транзитивно

ЧП обозначается символом $\preceq$

\subsubsection*{Примеры:}

\begin{enumerate}
	\item $\subseteq, U$
	\item $\le, \mathbb{R}$
\end{enumerate}

\subsubsection{Полный порядок}

$\rho$ на $X$ --- ПП, если $\rho$ --- ЧП, где
$\forall a, b \in X : a \rho b \lor b \rho a$

\pagebreak
\underline{Приведение к линейному порядку:}

\begin{gather*}
	X_1 = \min \limits_{X} \rho \\
	X_2 = \min \limits_{X \setminus X_1} \rho \\
	X_3 = \min \limits_{X \setminus (X_1 \cup X_2)} \rho \\
	\vdots \\
	X_k = \min \limits_{X \setminus (\bigcup \limits_{i=1}^{k-1} X_i)} \rho \\
	X_{k+1} = \emptyset
\end{gather*}

\subsubsection{Отношение эквивалентности}

$\rho$ на $X$ --- EQ, если $\rho$ --- рефлексивно, симметрично, транзитивно

\subsubsection{Класс эквивалентности}

\[
[x]_\rho = { y \in X : x \rho y }
\]

\subsection{Диаграммы Хассе}

\begin{gather*}
	x \prec y \Leftrightarrow x \preceq y \land x \ne y \\
	x \text{ покрывает } y \Leftrightarrow \not \exists u \in X : x \prec u \prec y
\end{gather*}

\section{Математическая логика}

\subsection{Логика высказываний}

\textbf{Высказывание} --- языковое предложение, о котором осмысленно говорить, истинно оно или ложно.

\textbf{Высказывательная перем.} --- буквенная замена высказывания, \\
принимающая значение истины или лжи. $x, y, z, x_1, \dots$

\subsubsection{Языковые средства}

Операции:

\begin{enumerate}
	\item Отрицание: "не". $\lnot x, \overline{x}$
	\item Конъюнкция: "и". $x \land y, x \& y$
	\item Дизъюнкция: "или". $x \land y, x | y$
	\item Импликация: "если то". $x \supset y, x \to y$
\end{enumerate}

\end{document}