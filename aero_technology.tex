\documentclass{article}
\usepackage[14pt]{extsizes}
\usepackage[T1, T2A]{fontenc}
\usepackage{indentfirst}
\usepackage[russian,english]{babel}
\usepackage{amsmath}
\usepackage{amssymb}
\usepackage[a4paper, margin=2.5cm]{geometry}

\begin{document}

\title{Дискретная математика}
\author{Илья Ковалев}
\date{2024 год}
\maketitle

\section{Информация}

Введение в авиационную и ракетно-космическую технику
Проф – Максим Юрьевич

\section{Организация проекта}

\begin{enumerate}
    \item Команда — 3/4 человека в пределах группы (до 16.09). Состав: \begin{itemize}
		\item Тим лид
		\item Математик/физик
		\item Программист
		\item Спикер (представление и оформление результатов)
	\end{itemize}
    \item Тема проекта
    \item Физическая модель — подобрать законы, удовлетворяющие задаче
    \item Математическая модель — уравнения
    \item Програмная реализация
    \item Валидация — Kerbal Space Program
    \item Представление — текстовый отчет, видео-отчет, презентация + \\
	доклад на 7 минут.
    \item Сдача — 15/16.12
\end{enumerate}

\section{Механика}

{\Large \underline{Механика:}}
\begin{itemize}
	\item \textbf{Как?} --- кинематика
	\item \textbf{Почему?} --- динамика
	\item \textbf{Равновесие?} --- статика
\end{itemize}

\subsection{Движение}

{\Large \underline{Движение:}}
\begin{enumerate}
	\item относительно
	\item продолжительно во времени
\end{enumerate}

\textbf{Материальная точка} --- тело, размером и формой которого можно пренебречь в пределах данной задачи.

\textbf{Поступательное движение} --- движение, при котором траектории всех точки параллельны друг другу.

При рассмотрении материальной точки,все движение является \\
поступательным.

\subsection{Уравнения движения}

$\vec{r}(t)$ --- закон движения

$\vec{r}(t = t_0) = \vec{r_0}$

$\vec{v} = \vec{r}_t^{\ '} = \dot{\vec{r}}$

$\Delta x$ --- приращение аргумента

$\Delta y$ --- приращение функции

\[y'(x) = \lim_{\Delta x \to 0} \frac{\Delta y}{\Delta x} = \frac{dy}{dx} \Rightarrow dy = y'dx\]

$\vec{a} = \dot{\vec{v}}$

\subsection{Производные}

\subsubsection*{В декартовой системе}

\begin{align*}
	dS &= dx dy \\
	dV &= dx dy dz
\end{align*}

\subsubsection*{В сферической системе}

\begin{align*}
	b &= r d\theta \\
	a &= r \sin \theta d\phi \\
	c &= dr
\end{align*}

\begin{align*}
	dS = ab &= r^2 \sin \theta d\theta d\phi \\
	dV = abc &= dS \cdot dr
\end{align*}

\begin{gather*}
	S = \int_{0}^{\pi} r^2 \sin \theta d\theta  \int_{0}^{2\pi} 1 d\phi = \\
	= r^2 \int_{0}^{2\pi} d\theta = r^2 \cdot 2\pi \cdot 2 \Rightarrow \\
	S = 4 \pi r^2
\end{gather*}

\begin{gather*}
	V = \int_{0}^{\pi} \int_{0}^{2\pi} \int_{0}^{R} r^2 \sin \theta d\theta d\phi dr = \\
	= \int_{0}^{\pi} \sin \theta d\theta \int_{0}^{2\pi} d\phi \int_{0}^{R} r^2 dr \Rightarrow \\
	V = \frac{4}{3} \pi R^3
\end{gather*}

\subsubsection*{В полярной системе}

\begin{align*}
	\vec{r} \rightarrow \vec{v} &= \frac{d\vec{r}}{dt} = \dot{\vec{r}}(t) \\
	\vec{v} \rightarrow \vec{a} &= \frac{d\vec{v}}{dt} = \frac{d^2\vec{r}}{dt^2}
\end{align*}

\begin{gather*}
	\begin{cases}
		x = r \cos \phi \\
		y = r \sin \phi
	\end{cases} \Rightarrow \\
	\begin{cases}
		v_x = \dot{x} = (r \cos \phi)_{t}^{'} = \dot{r} \cos \phi + r(\cos \phi)_{t}^{'} = 
		\dot{r} \cos \phi + r \sin \phi \cdot \dot{\phi} \\
		v_x = \dot{y} = (r \sin \phi)_{t}^{'} = \dot{r} \sin \phi + r(\sin \phi)_{t}^{'} =
		\dot{r} \sin \phi + r \cos \phi \cdot \dot{\phi}
	\end{cases}
\end{gather*}

\begin{gather*}
	\Delta t \to 0 \\
	v_1 = v_2 = v \\
	\vec{a} = \frac{\Delta \vec{v}}{\Delta t} \\
	\frac{R}{v} = \frac{S}{\Delta v} \Rightarrow \frac{R}{v} = \frac{v}{a} \Rightarrow a_n = \frac{v^2}{R}
\end{gather*}

\begin{gather*}
	a_\tau = \frac{d|\vec{v}|}{dt} \\
	\vec{a} = \vec{a}_n + \vec{a}_\tau
\end{gather*}

Полное ускорение --- сумма нормального ($a_n$) и тангенсального ($a_\tau$)

\begin{gather*}
	\omega = \dot{\phi} = \lim_{\Delta t \to 0} \frac{\Delta \phi}{\Delta t} \\
	\beta = \dot{\omega} =\ \stackrel{..}{\omega}\ = \lim_{\Delta t \to 0} \frac{\Delta \omega}{\Delta t}
\end{gather*}

\subsection{Векторное произведение}

\subsubsection*{Скалярное произведение}
\[ \vec{a} \cdot \vec{b} = (\vec{a}, \vec{b}) = (\vec{a} \cdot \vec{b}) \]

Пример: работа
\[ A = \vec{F} \cdot \Delta \vec{r} \]

\subsubsection*{Векторное произведение}
\[ \vec{a} \times \vec{b} = [\vec{a}, \vec{b}] = [\vec{a} \times \vec{b}] \]

Свойства:
\begin{align*}
	|\vec{a} \times \vec{b}| &= |\vec{a}| \cdot |\vec{b}| \sin \angle(\vec{a}, \vec{b}) \\
	\vec{a} \times \vec{b} &= -[\vec{a} \times \vec{b}]
\end{align*}

\end{document}