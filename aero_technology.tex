\documentclass{article}
\usepackage[14pt]{extsizes}
\usepackage[T1, T2A]{fontenc}
\usepackage{indentfirst}
\usepackage[russian,english]{babel}
\usepackage{amsmath}
\usepackage{amssymb}
\usepackage[a4paper, margin=2.5cm]{geometry}

\begin{document}

\title{Дискретная математика}
\author{Илья Ковалев}
\date{2024 год}
\maketitle

\section{Информация}

Введение в авиационную и ракетно-космическую технику
Проф – Максим Юрьевич

\section{Организация проекта}

\begin{enumerate}
    \item Команда — 3/4 человека в пределах группы (до 16.09). Состав: \begin{itemize}
		\item Тим лид
		\item Математик/физик
		\item Программист
		\item Спикер (представление и оформление результатов)
	\end{itemize}
    \item Тема проекта
    \item Физическая модель — подобрать законы, удовлетворяющие задаче
    \item Математическая модель — уравнения
    \item Програмная реализация
    \item Валидация — Kerbal Space Program
    \item Представление — текстовый отчет, видео-отчет, презентация + \\
	доклад на 7 минут.
    \item Сдача — 15/16.12
\end{enumerate}

\section{Механика}

{\Large \underline{Механика:}}
\begin{itemize}
	\item \textbf{Как?} --- кинематика
	\item \textbf{Почему?} --- динамика
	\item \textbf{Равновесие?} --- статика
\end{itemize}

\subsection{Движение}

{\Large \underline{Движение:}}
\begin{enumerate}
	\item относительно
	\item продолжительно во времени
\end{enumerate}

\textbf{Материальная точка} --- тело, размером и формой которого можно пренебречь в пределах данной задачи.

\textbf{Поступательное движение} --- движение, при котором траектории всех точки параллельны друг другу.

При рассмотрении материальной точки,все движение является \\
поступательным.

\subsection{Уравнения движения}

$\vec{r}(t)$ --- закон движения

$\vec{r}(t = t_0) = \vec{r_0}$

$\vec{v} = \vec{r}_t^{\ '} = \dot{\vec{r}}$

$\Delta x$ --- приращение аргумента

$\Delta y$ --- приращение функции

\[y'(x) = \lim_{\Delta x \to 0} \frac{\Delta y}{\Delta x} = \frac{dy}{dx} \Rightarrow dy = y'dx\]

$\vec{a} = \dot{\vec{v}}$

\end{document}