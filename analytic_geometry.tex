\documentclass{article}
\usepackage[14pt]{extsizes}
\usepackage[a4paper, margin=2.5cm]{geometry}
\usepackage{indentfirst}
\usepackage[T1, T2A]{fontenc}
\usepackage[russian,english]{babel}
\usepackage{amsmath}
\usepackage{amssymb}
\DeclareMathOperator*{\Rg}{Rg}
\DeclareMathOperator*{\Tr}{tr}

\begin{document}
\title{Аналитическая геометрия}
\author{Илья Ковалев}
\date{18 ноября 2024}
\maketitle

\section{Прямые}

\subsection{Общее уравнение прямой}

Прямая $l$ на плоскости $Oxy$

\begin{gather*}
	\begin{cases}
		M_0(x, y) \in l \\
		\vec{N} = \{ A, B \} \perp l
	\end{cases} \Rightarrow
	\forall M(x, y) \in l : \vec{M_0 M} \perp \vec{N} \Leftrightarrow \vec{N} \cdot \vec{M_0 M} = 0
\end{gather*}

\subsection{Параметрическое и каноническое уравнения \\
$l \subset Oxy$}

\begin{gather*}
	\begin{cases}
		M_0(x_0, y_0) \in l \\
		M(x, y) \\
		\vec{q} = \{ m, n \} \parallel l
	\end{cases} \Rightarrow \\
	\Rightarrow \forall M(x, y) \in l : \vec{M_0 M} \parallel \vec{q} \Leftrightarrow \\
	\Leftrightarrow \exists t \in \mathbb{R} : \vec{M_0 M} = t \vec{q} \Leftrightarrow \\
	\Leftrightarrow \begin{cases}
		x - x_0 = mt \\
		y - y_0 = nt
	\end{cases} \Leftrightarrow t = \frac{x - x_0}{m} = \frac{y - y0}{n} \Rightarrow \\
	\Rightarrow n(x - x_0) - m(y - y_0) = 0
\end{gather*}

\subsection{Прямая через 2 точки}

\begin{gather*}
	\begin{cases}
		M_1(x_1, y_1) \in l \\
		M_2(x_2, y_2) \in l
	\end{cases} \Rightarrow \frac{x - x_1}{x_2 - x_1} = \frac{y - y_1}{y_2 - y_1}
\end{gather*}

\subsection{Расстояние от точки до прямой}

\begin{gather*}
	\begin{cases}
		D(x_d, y_d) \\
		l: Ax + By + C = 0
	\end{cases} \Rrightarrow \rho = \frac{|Ax_d + By_d + C|}{\sqrt{A^2 + B^2}}
\end{gather*}

\end{document}
