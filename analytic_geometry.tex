\documentclass{article}
\usepackage[14pt]{extsizes}
\usepackage[a4paper, margin=2.5cm]{geometry}
\usepackage{indentfirst}
\usepackage[T1, T2A]{fontenc}
\usepackage[russian,english]{babel}
\usepackage{amsmath}
\usepackage{amssymb}
\DeclareMathOperator*{\Rg}{Rg}
\DeclareMathOperator*{\Tr}{tr}

\begin{document}
\title{Аналитическая геометрия}
\author{Илья Ковалев}
\date{18 ноября 2024}
\maketitle

\section{Прямая в плоскости}

\subsection{Общее уравнение прямой}

Прямая $l$ на плоскости $Oxy$

\begin{gather*}
	\begin{cases}
		M_0(x, y) \in l \\
		\vec{N} = \{ A, B \} \perp l
	\end{cases} \Rightarrow
	\forall M(x, y) \in l : \vec{M_0 M} \perp \vec{N} \Leftrightarrow \vec{N} \cdot \vec{M_0 M} = 0
\end{gather*}

\subsection{Параметрическое и каноническое уравнения \\
$l \subset Oxy$}

\begin{gather*}
	\begin{cases}
		M_0(x_0, y_0) \in l \\
		M(x, y) \\
		\vec{q} = \{ m, n \} \parallel l
	\end{cases} \Rightarrow \\
	\Rightarrow \forall M(x, y) \in l : \vec{M_0 M} \parallel \vec{q} \Leftrightarrow \\
	\Leftrightarrow \exists t \in \mathbb{R} : \vec{M_0 M} = t \vec{q} \Leftrightarrow \\
	\Leftrightarrow \begin{cases}
		x - x_0 = mt \\
		y - y_0 = nt
	\end{cases} \Leftrightarrow t = \frac{x - x_0}{m} = \frac{y - y_0}{n} \Rightarrow \\
	\Rightarrow n(x - x_0) - m(y - y_0) = 0
\end{gather*}

\subsection{Прямая через 2 точки}

\begin{gather*}
	\begin{cases}
		M_1(x_1, y_1) \in l \\
		M_2(x_2, y_2) \in l
	\end{cases} \Rightarrow \frac{x - x_1}{x_2 - x_1} = \frac{y - y_1}{y_2 - y_1}
\end{gather*}

\subsection{Расстояние от точки до прямой}

\begin{gather*}
	\begin{cases}
		D(x_d, y_d) \\
		l: Ax + By + C = 0
	\end{cases} \Rightarrow \rho = \frac{|Ax_d + By_d + C|}{\sqrt{A^2 + B^2}}
\end{gather*}

\section{Плоскость в пространстве}

\subsection{Общее уравнение плоскости}

\begin{gather*}
	\begin{cases}
		M_0(x_0, y_0, z_0) \in P \\
		\vec{N} = \{ A, B, C \} \perp P
	\end{cases} \Rightarrow \forall M(x, y, z) \in P : \vec{M_0 M} \perp \vec{N} \Leftrightarrow \\
	\Leftrightarrow A(x - x_0) + B(y - y_0) + C(z - z_0) = 0
\end{gather*}

\subsection{Компланарное уравнение}

\begin{gather*}
	\begin{cases}
		M(x_0, y_0, z_0) \in P \\
		\vec{p} \parallel P \\
		\vec{q} \parallel P \\
		\vec{p} \not \parallel \vec{q}
	\end{cases} \Rightarrow \forall M(x, y, z) \in P : \vec{M_0 M}, \vec{p}, \vec{q} \text{ --- компланарны} \Leftrightarrow \\
	\Leftrightarrow \begin{vmatrix}
		x - x_0 & y - y_0 & z - z_0 \\
		p_x & p_y & p_z \\
		q_x & q_y & q_z \\
	\end{vmatrix} = 0 \Rightarrow\\
	\Rightarrow (x - x_0) A_{11} + (y - y_0) A_{12} + (z - z_0) A_{13} = 0
\end{gather*}

\subsection{Через 3 точки}

\begin{gather*}
	\begin{cases}
		M_1(x_1, y_1, z_1) \in P \\
		M_2(x_2, y_2, z_2) \in P \\
		M_3(x_3, y_3, z_3) \in P \\
		M_1 \not \in \vec{M_1 M_2} \\
		M_2 \not \in \vec{M_2 M_3} \\
		M_3 \not \in \vec{M_3 M_1} \\
	\end{cases} \Leftrightarrow P : \begin{vmatrix}
		x - x_1 & y - y_1 & z - z_1 \\
		x_2 - x_1 & y_2 - y_1 & z_2 - z_1 \\
		x_3 - x_1 & y_3 - y_1 & z_3 - z_1 \\
	\end{vmatrix} = 0
\end{gather*}

\subsection{Расстояние от точки до плоскости}

\begin{gather*}
	\begin{cases}
		P(x_0, y_0, z_0) \\
		Ax + By + Cz + D = 0
	\end{cases} \Rightarrow
	\rho = \frac{|Ax_0 + By_0 + Cz_0 + D|}{\sqrt{A^2 + B^2 + C^2}}
\end{gather*}

\subsection{Взаимное расположение 2 плоскостей}

\[
\begin{cases}
	P1 : A_1x + B_1y + C_1z + D_1 = 0 \\
	P2 : A_2x + B_2y + C_2z + D_2 = 0 \\
\end{cases}
\]

\subsubsection{Параллельные}

\begin{gather*}
	P_1 \parallel P_2 \Leftrightarrow \frac{A_1}{A_2} =
	\frac{B_1}{B_2} = \frac{C_1}{C_2} \\
	\rho(P_1 \parallel P_2) = \rho(M_1 \in P_1, P_2) =
	\frac{|D_2 C_1 - D_1 C_2|}{|C_1| \sqrt{A_2^2 + B_2^2 + C_2^2}}
\end{gather*}

\subsubsection{Пересекающиеся}

\begin{gather*}
	P_1 \not \parallel P_2 \Leftrightarrow
	\frac{A_1}{A_2} \ne \frac{B_1}{B_2} \lor \frac{B_1}{B_2} \ne \frac{C_1}{C_2} \\
	\alpha = \angle(P_1, P_2) \in (0; \frac{\pi}{2}] :
	\cos \alpha = |\cos \phi| = \frac{|\vec{N_1} \cdot \vec{N_2}|}{|\vec{N_1}| |\vec{N_2}|}
\end{gather*}

\section{Прямая в пространстве}

\subsection{Общее уравнение}

Пересечение плоскостей

\begin{gather*}
	L : \begin{cases}
		A_1x + B_1y + C_1z + D_1 = 0 \\
		A_2x + B_2y + C_2z + D_2 = 0 \\
		\frac{A_1}{A_2} \ne \frac{B_1}{B_2} \lor \frac{B_1}{B_2} \ne \frac{C_1}{C_2}
	\end{cases} \\
	\vec{N_1} = \{ A_1, B_1, C_1 \} \\
	\vec{N_2} = \{ A_2, B_2, C_2 \} \\
\end{gather*}

\subsection{Параметрическое и каноническое}

\begin{gather*}
	\begin{cases}
		M_0(x_0, y_0, z_0) \in L \\
		\vec{q} = \{ m, n, p \} \parallel L
	\end{cases} \Rightarrow \\
	\Rightarrow \begin{cases}
		x - x_0 = mt \\
		y - y_0 = nt \\
		z - z_0 = pt \\
		m^2 + n^2 + p^2 \ne 0
	\end{cases} \Leftrightarrow \\
	\Leftrightarrow t = \frac{x - x_0}{m} = \frac{y - y_0}{n} = \frac{z - z_0}{p} \\
	\vec{q} = \vec{N_1} \times \vec{N_2} \\
	M_0 : \begin{cases}
		A_1x_0 \dots = 0 \\
		A_2x_0 \dots = 0 \\
		x_0 = 0 \lor y_0 = 0 \lor z_0 = 0
	\end{cases} \rightarrow M_0(x_0, y_0, z_0)
\end{gather*}

\subsection{Расстояние от точки до прямой}

\begin{gather*}
	\begin{cases}
		M_1(x_1, y_1, z_1) \\
		L: t = \frac{x - x_0}{m} = \frac{y - y_0}{n} = \frac{z - z_0}{p} \\
	\end{cases} \Rightarrow
	\rho(M_1, L) = \frac{|\vec{M_0 M_1} \times \vec{q}|}{|\vec{q}|}
\end{gather*}

\subsection{Взаимное расположение 2 прямых}

\[
	\begin{cases}
		L_1: \frac{x - x_1}{m_1} = \frac{y - y_1}{n_1} = \frac{z - z_1}{p_1} \\
		L_2: \frac{x - x_2}{m_2} = \frac{y - y_2}{n_2} = \frac{z - z_2}{p_2}
	\end{cases}
\]

\subsubsection{Параллельные}

\begin{gather*}
	\frac{m_1}{m_2} = \frac{n_1}{n_2} = \frac{p_1}{p_2} \Rightarrow L_1 \parallel L_2 \\
	\rho(L_1, L_2) = \rho(M_1 \in L_1, L_2) = \frac{|\vec{M_2 M_1} \times \vec{q_2}|}{|\vec{q_2}|}
\end{gather*}

\subsubsection{Непараллельные}

\begin{gather*}
	\begin{cases}
		\frac{m_1}{m_2} \ne \frac{n_1}{n_2} \lor \frac{n_1}{n_2} \ne \frac{p_1}{p_2} \\
		\rho(L_1, L_2) = \frac{|(\vec{M_1 M_2}, \vec{q_1}, \vec{q_2})|}{|\vec{q_1} \times \vec{q_2}|}
	\end{cases} \\
	\alpha = \angle(L_1, L_2) \in [0; \frac{\pi}{2}] \\
	\cos \alpha = |\cos \phi| = \frac{|\vec{q_1} \cdot \vec{q_2}|}{|\vec{q_1}| \cdot |\vec{q_2}|}
\end{gather*}

$\rho(L_1, L_2) = 0$ --- пересекающиеся

$\rho(L_1, L_2) \ne 0$ --- скрещивающиеся

\subsection{Взаимное расположение прямой и плоскости}

\[
	\begin{cases}
		L: \frac{x - x_1}{m_1} = \frac{y - y_1}{n_1} = \frac{z - z_1}{p_1} \\
		P: Ax + By + Cz + D = 0 \rightarrow \vec{N} = \{ A, B, C \} \perp P
	\end{cases}
\]

\subsubsection{Параллельные}

\[
	L \parallel P \Leftrightarrow
	\vec{q} \perp \vec{N} \Leftrightarrow
	\vec{q} \cdot \vec{N} \Leftrightarrow
	Am + Bn + Cp = 0
\]

\subsubsection{Пересекающиеся}

\begin{gather*}
	L \not \parallel P \Leftrightarrow Am + Bn + Cp \ne 0 \\
	\beta = \angle(L, P) \\
	\sin \beta = |\cos \phi| = \frac{|\vec{N} \cdot \vec{q}|}{|\vec{N}| \cdot |\vec{q}|} \\
	K = L \cap P : \begin{cases}
		\frac{x - x_0}{m} = \frac{y - y_0}{n} = \frac{z - z_0}{p} = t \\
		Ax + By + Cz + D = 0
	\end{cases} \Rightarrow \begin{cases}
		x = x_0 + mt \\
		y = y_0 + nt \\
		z = z_0 + pt \\
	\end{cases}
\end{gather*}

\end{document}
