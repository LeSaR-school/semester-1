\documentclass{article}
\usepackage[14pt]{extsizes}
\usepackage[T1, T2A]{fontenc}
\usepackage{indentfirst}
\usepackage[russian,english]{babel}
\usepackage{amsmath}
\usepackage{amssymb}
\usepackage[a4paper, margin=2.5cm]{geometry}

\begin{document}

\title{Архитектура компьютера и информационных систем}
\author{Илья Ковалев}
\date{2024 год}
\maketitle

\section{Представление вещественных чисел}

\subsection{Хранение}

Вещественные числа в компьютере хранятся в одном "слове" --- 64 битах (на современных системах)

\subsection{Запись числа в 64 битных системах}

\begin{tabular}{|c||c|c|c||c|c|c|}
	\hline
	\text{знак} & $e_1$ & $e_2 \dots e_{10}$ & $e_{11}$ & $v_1$ & $v_2 \dots v_{51}$ & $v_{52}$ \\
	\hline
\end{tabular}

Где $e_{0 \dots i}$ --- экспонент, $v_{0 \dots i}$ --- значение

\section{Обработка сообщений}

\subsection{Понятия}

$I$ --- множество информации

$N$ --- множество сообщений

$\phi: N \rightarrow I$ --- функция интерпретации

$\nu: N \rightarrow N'$ --- функция преображения сообщений

$\rho: I \rightarrow I'$ --- функция преображения информации

$D$ --- представление в компьютере

$C$ --- функция превращения сообщений в их компьютерные \\
представления

$P$ --- функция обработки данных

$Q$ --- функция раскодирования компьютерных представлений

$\nu = Q \circ P \circ C$

\begin{alignat*}{5}
	D & \stackrel{C}{\longleftarrow} & N & \stackrel{\phi}{\longrightarrow} & I \\
	P \downarrow & & \nu \downarrow & & \rho \downarrow \\
	D' & \stackrel{Q}{\longrightarrow} & N' & \stackrel{\phi'}{\longrightarrow} & I'
\end{alignat*}

\subsection{Свойства алгоритма}

\begin{itemize}
	\item Массовость
	\item Детерминированность
	\item Элементарность
	\item Результативность
\end{itemize}

\subsection{Сложность}

Пусть $f(n) = O(g(n))$

Тогда $\exists C$--- const$, n_0 \in \mathbb{N} : \forall n > n_0 \Rightarrow f(n) \le C g(n)$

\subsection{Моделирование}

Рассмотрим две МТ: $T = (A, Q, P, q_0)$ и $T' = (A', Q', P', q_0')$.

Тогда $T'$ моделирует $T$, если:

\begin{enumerate}
	\item Указан способ кодирования $C: A \rightarrow A'$
	\item Каждому состоянию $q \in Q$ машины $T$ поставлено в соответствие некоторое 
	состояние $q' \in Q'$, то есть определено отражение $Q \rightarrow Q'$
	\item Если $C_0$ --- начальная конфигурация $T$, то $C_0'$ --- начальная \\
	конфигурация $T'$
	\item Если $T$ из $C_0$ после конечного числа тактов останавливается в \\
	конфигурации $C_1$, то и $T'$ из $C_0'$ после конечного числа тактов \\
	останавливается в конфигурации $C_1'$
\end{enumerate}

\end{document}
