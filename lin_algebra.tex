\documentclass{article}
\usepackage[14pt]{extsizes}
\usepackage[a4paper, margin=2.5cm]{geometry}
\usepackage{indentfirst}
\usepackage[T1, T2A]{fontenc}
\usepackage[russian,english]{babel}
\usepackage{amsmath}

\begin{document}
\title{Линейная алгебра и аналитическая геометрия}
\author{Илья Ковалев}
\date{6 сентября 2024}
\maketitle

\section{Экзамен}

Билет --- 5 вопросов, 2 теория и 3 практика

1 вопрос = 1 балл

Письменный экзамен, длительность --- 90 минут

\section{Матрицы и их операции}

Матрицей $A$ порядка $m * n$ называют двумерную таблицу, состоящую из $m$ строк и $n$ столбцов.

Прямоугольная матрица --- $m * n$.

Квадратная матрица --- $n * n$.

Диагональная матрица --- $n * n$, где отличны от нуля только элементы главной диагонали.

\[
\begin{pmatrix}
	a & 0 & 0 & 0\\
	0 & b & 0 & 0\\
	0 & 0 & c & 0\\
	0 & 0 & 0 & d
\end{pmatrix}
\]

Скалярная --- диагональная, где все элементы диагонали равны.

\[
\begin{pmatrix}
	a & 0 & 0\\
	0 & a & 0\\
	0 & 0 & a
\end{pmatrix}
\]

Единицная --- скалярная, где все элементы диагонали $= 1$.

\[
\begin{pmatrix}
	1 & 0 & 0\\
	0 & 1 & 0\\
	0 & 0 & 1
\end{pmatrix}
\]

Нулевая --- все элементы $= 0$.

\[
\begin{pmatrix}
	0 & 0 \\
	0 & 0
\end{pmatrix}
\]

\section{Вычисление определителей}

$detA_{n*m}$

\subsection{Младшие порядки}
\[
n = 2
\]

\[
\begin{vmatrix}
	a & b \\
	c & d
\end{vmatrix}
= ad - bc
\]

\[
\begin{vmatrix}
	-5 & -6 \\
	-7 & -8
\end{vmatrix}
= (-5)(-8) - (6)(-7) = 40 + 42 = 82
\]

\subsubsection{Способ 1. По Саррюсу}
\[
n = 3
\]

\subsubsection*{№2.13}

\begin{gather*}
\begin{vmatrix}
	3 & 4 & -5 \\
	8 & 7 & -2 \\
	2 & -1 & 8 \\	
\end{vmatrix}
\begin{matrix}
	3 & 4 \\
	8 & 7 \\
	2 & -1
\end{matrix} = \\
= (3)(7)(8) + (4)(-2)(2) + (-5)(8)(-1) - \\
- (2)(7)(-5) - (-1)(-2)(3) - (8)(8)(4) = \\
168 - 16 + 40 + 70 - 6 - 256 = 0
\end{gather*}

\subsubsection{Способ 2. Разложение}

\[
\begin{vmatrix}
	a & b & c \\
	d & e & f \\
	m & n & k	
\end{vmatrix} =
\sum_{i=1}^{3} a_{i2} A_{i2} =
bA_{12} + eA_{22} + nA_{32}
\]

\begin{gather*}
	\begin{vmatrix}
		3 & 4 & -5 \\
		8 & 7 & -2 \\
		2 & -1 & 8
	\end{vmatrix} = \sum_{i=1}^{3} a_{i2} A_{i2} = \\
	= 4A_{12} + 7A_{22} + (-1)A_{32} =
	4(-M_{12}) + 7M_{22} + M_{32}		
\end{gather*}

где

\begin{align*}
	M_{12} &= 
	\begin{vmatrix}
		8 & -2 \\
		2 & 8
	\end{vmatrix}
	= 64 + 4 = 68 \\
	M_{22} &=
	\begin{vmatrix}
		3 & -5 \\
		2 & 8
	\end{vmatrix}
	= 24 + 10 = 34 \\
	M_{32} &=
	\begin{vmatrix}
		3 & -5 \\
		8 & -2
	\end{vmatrix}
	= -6 + 40 = 34 \\
\end{align*}

\[
	4(-68) + 7(34) + 34 = (-8 + 7 + 1) * 34 = 0
\]

\subsubsection{Контроль}

\begin{gather*}
	\begin{vmatrix}
		3 & 2 & 1 \\
		2 & 5 & 3 \\
		3 & 4 & 2
	\end{vmatrix}
	\begin{matrix}
		3 & 2 \\
		2 & 5 \\
		3 & 4
	\end{matrix} = \\
	= (3)(5)(2) + (2)(3)(3) + (1)(2)(4) - \\
	- (3)(5)(1) - (4)(3)(3) - (2)(2)(2) = \\
	= 30 + 18 + 8 - 15 - 36 - 8 = -3
\end{gather*}

\subsubsection*{№2.54(a)}

\[
	\begin{vmatrix}
	2 & -3 & 4 & 1 \\
	4 & -2 & 3 & 2 \\
	a & b & c & d \\
		3 & -1 & 4 & 3
	\end{vmatrix}
	= a A_{31} + b A_{32} + c A_{33} + d A_{34}
\]

\begin{gather*}
	A_{31} = (-1)^{3+1} M_{31} = \\
	= \begin{vmatrix}
		-3 & 4 & 1 \\
		-2 & 3 & 2 \\
		-1 & 4 & 3
	\end{vmatrix}
	\begin{matrix}
		-3 & 4 \\
		-2 & 3 \\
		-1 & 4
	\end{matrix} = \\
	= (-3)(3)(3) + (4)(2)(-1) + (1)(-2)(4) - \\
	- (-1)(3)(1) - (4)(2)(-3) - (3)(-2)(4) = \\
	= -27 - 8 - 8 + 3 + 24 + 24 = 8
\end{gather*}

\begin{gather*}
	A_{32} = (-1)^{3+2} M_{32} = \\
	= -\begin{vmatrix}
		2 & 4 & 1 \\
		4 & 3 & 2 \\
		3 & 4 & 3
	\end{vmatrix}
	\begin{matrix}
		2 & 4 \\
		4 & 3 \\
		3 & 4
	\end{matrix} = \\
	= -(2)(3)(3) - (4)(2)(3) - (1)(4)(4) + \\
	+ (3)(3)(1) + (4)(2)(2) + (3)(4)(4) = \\
	= -(18 + 24 + 16 - 9 - 16 - 48) = 15	
\end{gather*}

\begin{gather*}
	A_{33} = (-1)^{3+3} M_{33} = \\
	= \begin{vmatrix}
		2 & -3 & 1 \\
		4 & -2 & 2 \\
		3 & -1 & 3
	\end{vmatrix}
	\begin{matrix}
		2 & -3 \\
		4 & -2 \\
		3 & -1
	\end{matrix} = \\
	= (2)(-2)(3) + (-3)(2)(3) + (1)(4)(-1) - \\
	- (3)(-2)(1) - (-1)(2)(2) - (3)(4)(-3) = \dots
\end{gather*}

\subsubsection{Способ 3. С упрощением}

\begin{gather*}
	\begin{vmatrix}
		2 & -1 & 1 & 0 \\
		0 & 1 & 2 & -1 \\
		3 & -1 & 2 & 3 \\
		3 & 1 & 6 & 1
	\end{vmatrix}
	\begin{matrix}
		I \\
		II + IV \\
		III - 3IV \\
		IV
	\end{matrix} = \\
	= \begin{vmatrix}
		2 & -1 & 1 & 0 \\
		3 & 2 & 8 & 0 \\
		-6 & -4 & -16 & 0 \\
		3 & 1 & 6 & 1
	\end{vmatrix} = \\
	= 0 A_{14} + 0 A_{24} + 0 A_{34} + 1 A_{44} =
	M_{44} = \\
	= \begin{vmatrix}
		2 & -1 & 1 \\
		3 & 2 & 8 \\
		-6 & -4 & -16
	\end{vmatrix}
	= -2 * \begin{vmatrix}
		2 & -1 & 1 \\
		3 & 2 & 8 \\
		3 & 2 & 8
	\end{vmatrix}
	= -2 * 0 = 0
\end{gather*}

\subsubsection{Контроль}

\subsubsection*{№2.56}

\begin{gather*}
	\begin{vmatrix}
		2 & 3 & -3 & 4 \\
		2 & 1 & -1 & 2 \\
		6 & 2 & 1 & 0 \\
		2 & 3 & 0 & -5
	\end{vmatrix}
	\begin{matrix}
		I - 3II \\
		II \\
		III + II \\
		IV
	\end{matrix} = \\
	= \begin{vmatrix}
		-4 & 0 & 0 & -2 \\
		2 & 1 & -1 & 2 \\
		8 & 3 & 0 & 2 \\
		2 & 3 & 0 & 5
	\end{vmatrix} = \\
	= -A_{23} = M_{23} = \begin{vmatrix}
		-4 & 0 & -2 \\
		8 & 3 & 2 \\
		2 & 3 & -5
	\end{vmatrix} = \\
	= \begin{vmatrix}
		-4 & 0 & -2 \\
		6 & 0 & 7 \\
		2 & 3 & -5
	\end{vmatrix}
	= -3
	\begin{vmatrix}
		-4 & -2 \\
		6 & 7
	\end{vmatrix} = \\
	= -3 * (-28) + 12 = -3 * (-16) = 48
\end{gather*}

\subsubsection{ДЗ}

№№
\begin{itemize}
\item{2.1}
\item{.50}
\item{.52}
\item{.54(б)}
\item{.57}
\item{61*}
\end{itemize}

\subsection{Свойства определителей}

\begin{enumerate}
	\item Транспонирование --- строки и столбцы равноправны
	\item Упрощение
	\item Перестановка двух строк/столбцов меняет знак определителя, не меняет модуль
	\item Умножение
	\[
	\begin{vmatrix}
		1 & 2 & 3 \\
		4 & 5 & 6 \\
		4 & 8 & 12
	\end{vmatrix} =
	4 \begin{vmatrix}
		1 & 2 & 3 \\
		4 & 5 & 6 \\
		1 & 2 & 3
	\end{vmatrix}
	\]
	\item Сложение
	\[
	\begin{vmatrix}
		1 & 2 & 3 \\
		4 & 5 & 6 \\
		7 & 8 & 9
	\end{vmatrix} +
	\begin{vmatrix}
		1 & 2 & 3 \\
		-1 & 3 & -4 \\
		7 & 8 & 9
	\end{vmatrix} =
	4 \begin{vmatrix}
		1 & 2 & 3 \\
		3 & 8 & 2 \\
		7 & 8 & 9
	\end{vmatrix}
	\]
	\item Спец-свойство
	\[
	i \ne j, \sum_{k=1}^{n} a_{ik} A_{jk} = 0
	\]
	\item Произведение
	\[
	det A * det B = det A * B
	\]
	\item Треугольный определитель
	\[
	det U_{4*\times 4} =
	\begin{vmatrix}
		a_{11} & a_{12} & a_{13} & a_{14} \\
		0 & a_{22} & a_{23} & a_{24} \\
		0 & 0 & a_{33} & a_{34} \\
		0 & 0 & 0 & a_{44} \\
	\end{vmatrix} =
	a_{11} * a_{22} * a_{33} * a_{44}
	\]
\end{enumerate}

\pagebreak
\section{Обратная матрица}

Деление в алгебре:

$ax = b \stackrel{a \ne 0}{\Rightarrow} x = \frac{b}{a}$

\textbf{Обратное число} для $a \ne 0$ --- такое $a^{-1}$, что $a * a^{-1} = 1$

\subsection{Определение}

Обратная матрица $A^{-1}$ --- такая, что ее произведение и слева, и справа --- единичная матрица.

\begin{align*}
A A^{-1} &\stackrel{\triangle}{=} E \\
A^{-1} A &\stackrel{\triangle}{=} E
\end{align*}

\subsection{Свойства}

\begin{enumerate}
	\item \textbf{Порядок} --- $A_{n \times n} \Rightarrow A_{n \times n}^{-1}$
	\item \textbf{Единственность} --- $A : \exists A^{-1}, \exists ! A^{-1}$\\
	Доказательство от противного:\\
	Предположим, что $A : \exists A^{-1}_1 \ne A^{-1}_2$\\
	Тогда $A A_1^{-1} - A A_2^{-1} = E - E = 0$\\
	$A(A_1^{-1} - A_2^{-1}) = 0$\\
	$A_1^{-1} - A_2^{-2} = 0 \Rightarrow A_1^{-1} = A_2^{-1}$ --- противоречие. ЧТД.
	\item \textbf{Обратность определителя} --- $A : \exists A^{-1} \Rightarrow det A = \frac{1}{det A^{-1}}$\\
	$A A^{-1} = E$\\
	$det A A^{-1} = det E = 1$\\
	$det A = \frac{1}{det A^{-1}}$
	\item \textbf{Ненулевость определителя} --- $det A = 0 \Rightarrow \not \exists A^{-1}$\\
	Если $A : \exists A^{-1}$, то $det A \ne 0$
	\item $A : det A \ne 0 \Rightarrow A^{-1} = \frac{A^*}{det A}$\\
	$A^* = adj(A) = (A_{ij})_{i,j = T,i}^T$
\end{enumerate}

\textbf{Вырожденная матрица} --- $A, det A = 0$

\subsection{Алгоритм обращения матрицы}

\begin{enumerate}
	\item $A^{-1} = \frac{A_{ij}^T}{det A}$
	\item Вычислить все $A_{ij}$ для $a_{ij}$.
	\item Собрать все $A_{ij}$ в матрицу и транспонировать ее.
	\[
	A^* = \begin{pmatrix}
		A_{11} & A_{12} & \dots & A_{1n} \\
		A_{21} & A_{22} & \dots & A_{2n} \\
		\vdots & \vdots & & \vdots \\
		A_{m1} & A_{m2} & \dots & A_{mn}
	\end{pmatrix}^T =
	\begin{pmatrix}
		A_{11} & A_{21} & \dots & A_{m1} \\
		A_{12} & A_{22} & \dots & A_{m2} \\
		\vdots & \vdots & & \vdots \\
		A_{1n} & A_{2n} & \dots & A_{mn}
	\end{pmatrix}
	\]
	\item Разделить. $A^{-1} = \frac{A^*}{det A}$
	\item Проверить. $A A^{-1} = E$
\end{enumerate}

\subsubsection{Пример --- быстрое обращение матрицы $2 \times 2$}

$
n = 2 : A = \begin{pmatrix}
	a & b \\
	c & d
\end{pmatrix}
$
\begin{enumerate}
	\item Пусть $det A = ad - bc \ne 0$
	\item Найдем все алгебраические дополнения 
	\begin{gather*}
		A_{11} = M_{11} = d \\
		A_{12} = -M_{12} = -c \\
		A_{21} = -M_{21} = -b \\
		A_{22} = M_{22} = a \\
	\end{gather*}
	\item \[ A^* = \begin{pmatrix}
		d & -c \\
		-b & a
	\end{pmatrix}^T =
	\begin{pmatrix}
		d & -b \\
		-c & a
	\end{pmatrix} \]
	\item \[ A^{-1} = \frac{A^*}{det A} \]
\end{enumerate}

\subsection{Свойства операций обращения матриц}

Примечание: $\forall A \exists A^{-1}$

\begin{enumerate}
	\item $(AB)^{-1} = A^{-1} B^{-1}$
	\item $(\lambda B)^{-1} = \lambda^{-1} B^{-1}$
	\item $(A^{-1})^{-1} = A$
	\item $(A^{-1})^T = (A^T)^{-1} = A^{-T}$
\end{enumerate}

\subsection{Примеры}

\subsubsection*{№2.106}

\begin{gather*}
	n = 2 \\
	A = \begin{pmatrix}
		1 & 2 \\
		3 & 4
	\end{pmatrix} \\
	A^{-1} = \frac{1}{4 - 6} \begin{pmatrix}
		4 & -2 \\
		-3 & 1
	\end{pmatrix} \\
	A^{-1} = \begin{pmatrix}
		-2 & 1 \\
		1.5 & -0.5
	\end{pmatrix}
\end{gather*}

Проверка:

\begin{gather*}
	A A^{-1} = \begin{pmatrix}
		(1)(-2) + (2)(1.5) & (-2)(2) + (1)(4) \\
		(3)(-2) + (4)(1.5) & (1.5)(2) + (4)(-0.5)
	\end{pmatrix} = \\
	= \begin{pmatrix}
		1 & 0 \\
		0 & 1
	\end{pmatrix} = E_{2 \times 2}
\end{gather*}

\subsubsection*{№2.109}

\begin{enumerate}
	\item Определитель
	\begin{gather*}
		n = 3 \\
		A = \begin{pmatrix}
			2 & 5 & 7 \\
			6 & 3 & 4 \\
			5 & -2 & -3
		\end{pmatrix} \\
		det A = \begin{vmatrix}
			2 & 5 & 7 \\
			6 & 3 & 4 \\
			5 & -2 & -3
		\end{vmatrix} = \dots = -1 \ne 0 \Rightarrow \exists A^{-1}
	\end{gather*}
	\item Первая строка
	\begin{gather*}
		A_{11} = M_{11} = \begin{vmatrix}
			3 & 4 \\
			-2 & -3
		\end{vmatrix} = -9 + 8 = -1 \\
		A_{12} = -M_{12} = -\begin{vmatrix}
			6 & 4 \\
			5 & -3
		\end{vmatrix} = -(-18 + (-20)) = 38 \\
		A_{13} = M_{13} = \begin{vmatrix}
			6 & 3 \\
			5 & -2
		\end{vmatrix} = -12 - 15 = -27
	\end{gather*}
	\item Вторая строка
	\begin{gather*}
		A_{21} = -M_{21} = \begin{vmatrix}
			5 & 7 \\
			-2 & -3
		\end{vmatrix} = -(-15 + 14) = 1 \\
		A_{22} = M_{22} = -\begin{vmatrix}
			2 & 7 \\
			5 & -3
		\end{vmatrix} = -6 + (-35) = -41 \\
		A_{23} = -M_{23} = \begin{vmatrix}
			2 & 5 \\
			5 & -2
		\end{vmatrix} = -(-4 - 25) = 29
	\end{gather*}
	\item Третья строка
	\begin{gather*}
		A_{31} = M_{31} = \begin{vmatrix}
			5 & 7 \\
			3 & 4
		\end{vmatrix} = 20 - 21 = -1 \\
		A_{32} = -M_{32} = -\begin{vmatrix}
			2 & 7 \\
			6 & 4
		\end{vmatrix} = -(8 - 42) = 34 \\
		A_{33} = M_{33} = \begin{vmatrix}
			2 & 5 \\
			6 & 3
		\end{vmatrix} = 6 - 30 = -24
	\end{gather*}
	\item Присоединенная матрица
	\begin{gather*}
		A^* = \begin{pmatrix}
			A_11 & A_21 & A_31 \\
			A_12 & A_22 & A_32 \\
			A_13 & A_23 & A_33
		\end{pmatrix} = \begin{pmatrix}
			-1 & 1 & -1 \\
			38 & -41 & 34 \\
			-27 & 29 & -24
		\end{pmatrix}
	\end{gather*}
	\item Обратная матрица
	\begin{gather*}
		A^{-1} = \frac{A^*}{-1} = \begin{pmatrix}
			1 & -1 & 1 \\
			-38 & 41 & -34 \\
			27 & -29 & 24
		\end{pmatrix}
	\end{gather*}
\end{enumerate}




\pagebreak
\section{Решение линейных уравнений}

\subsubsection*{Тип 1}

\begin{align*}
A_{m \times n} X_{n \times P} &= B_{m \times P}\\
A_{n \times m}^{-1} A_{m \times n} X_{n \times P} &= A_{n \times m}^{-1} B_{m \times P}\\
X_{n \times P} &= A_{n \times m}^{-1} B_{m \times P}
\end{align*}

\subsubsection*{Тип 2}

\begin{align*}
Y A &= B \\
Y A A^{-1} &= B A^{-1} \\
Y &= B A^{-1}
\end{align*}

\subsubsection*{Тип 3}

\begin{align*}
	A_1 X A_2 &= B \\
	A_1^{-1} A_1 X A_2 A_2^{-1} &= A_1^{-1} B A_2^{-1} \\
	Y &= A_1^{-1} B A_2^{-1}
\end{align*}	

\subsubsection*{Тип 4}

$A_1 X + X A_2 = B$ --- не вычисляется с помощью обратных матриц

\subsection{Значения матриц}

$
X = \begin{pmatrix}
	x_1 \\
	x_2 \\
	\vdots \\
	x_n
\end{pmatrix}
$ --- переменные

$B = \begin{pmatrix}
	b_1 \\
	b_2 \\
	\vdots \\
	b_m
\end{pmatrix}$ --- свободные члены

$A = \begin{pmatrix}
	a_{11} & a_{12} & \dots & a_{1n} \\
	a_{21} & a_{22} & \dots & a_{2n} \\
	\vdots & \vdots & & \vdots \\
	a_{m1} & a_{m2} & \dots & a_{mn} \\
\end{pmatrix}$ --- коэффициенты

\subsection{Примеры}

\subsubsection*{№2.121}

\begin{enumerate}
	\item \[
	\begin{pmatrix}
		1 & 2 \\
		3 & 4
	\end{pmatrix} X =
	\begin{pmatrix}
		3 & 5 \\
		5 & 9
	\end{pmatrix}
	\]
	\item \begin{gather*}
		X = A^{-1} B = \\
		= -\frac{1}{2}
		\begin{pmatrix}
			4 & -2 \\
			-3 & 1
		\end{pmatrix}
		\begin{pmatrix}
			3 & 5 \\
			5 & 9
		\end{pmatrix} = \\
		= -\frac{1}{2}
		\begin{pmatrix}
			2 & 2 \\
			-4 & -6
		\end{pmatrix} = \\
		\begin{pmatrix}
			-1 & -1 \\
			2 & 3
		\end{pmatrix}
	\end{gather*}
	\item Проверка
	\begin{gather*}
		A X \stackrel{?}{=} B \\
		A X = \begin{pmatrix}
			1 & 2 \\
			3 & 4
		\end{pmatrix}
		\begin{pmatrix}
			-1 & -1 \\
			2 & 3
		\end{pmatrix}
		= \begin{pmatrix}
			3 & 5 \\
			5 & 9
		\end{pmatrix}
	\end{gather*}
	Верно
\end{enumerate}

\subsubsection*{№3.122}

\[
X \begin{pmatrix}
	3 & -2 \\
	5 & -4
\end{pmatrix} =
\begin{pmatrix}
	-1 & 2 \\
	-5 & 6
\end{pmatrix}
\]

\begin{enumerate}
	\item \[
	A^{-1} = \frac{1}{det A} \begin{pmatrix}
		-4 & 2 \\
		-5 & 3
	\end{pmatrix} =
	\frac{1}{-2} \begin{pmatrix}
		-4 & 2 \\
		-5 & 3
	\end{pmatrix}
	\]
	\item $X = \dots$ --- мне лень
\end{enumerate}

\subsubsection{ДЗ}

№№
\begin{enumerate}
	\item 2.107
	\item .110
	\item .123
	\item .125
\end{enumerate}

\end{document}
