\documentclass{article}
\usepackage[14pt]{extsizes}
\usepackage[a4paper, margin=2.5cm]{geometry}
\usepackage{indentfirst}
\usepackage[T1, T2A]{fontenc}
\usepackage[russian,english]{babel}
\usepackage{amsmath}

\begin{document}
\title{Линейная алгебра и аналитическая геометрия}
\author{Илья Ковалев}
\date{6 сентября 2024}
\maketitle

\section{Экзамен}

Билет --- 5 вопросов, 2 теория и 3 практика

1 вопрос = 1 балл

Письменный экзамен, длительность --- 90 минут

\section{Матрицы и их операции}

Матрицей $A$ порядка $m * n$ называют двумерную таблицу, состоящую из $m$ строк и $n$ столбцов.

Прямоугольная матрица --- $m * n$.

Квадратная матрица --- $n * n$.

Диагональная матрица --- $n * n$, где отличны от нуля только элементы главной диагонали.

$
\begin{pmatrix}
	a & 0 & 0 & 0\\
	0 & b & 0 & 0\\
	0 & 0 & c & 0\\
	0 & 0 & 0 & d
\end{pmatrix}
$

Скалярная --- диагональная, где все элементы диагонали равны.

$
\begin{pmatrix}
	a & 0 & 0\\
	0 & a & 0\\
	0 & 0 & a
\end{pmatrix}
$

Единицная --- скалярная, где все элементы диагонали $= 1$.

$
\begin{pmatrix}
	1 & 0 & 0\\
	0 & 1 & 0\\
	0 & 0 & 1
\end{pmatrix}
$

Нулевая --- все элементы $= 0$.

$
\begin{pmatrix}
	0 & 0 \\
	0 & 0
\end{pmatrix}
$

\section{Вычисление определителей}

$detA_{n*m}$

\subsection{Младшие порядки}
$
n = 2
$

$
\begin{vmatrix}
	a & b \\
	c & d
\end{vmatrix}
= ad - bc
$

$
\begin{vmatrix}
	-5 & -6 \\
	-7 & -8
\end{vmatrix}
= (-5)(-8) - (6)(-7) = 40 + 42 = 82
$

\subsubsection{Способ 1. По Саррюсу}
$
n = 3
$

№2.13

$
\begin{vmatrix}
	3 & 4 & -5 \\
	8 & 7 & -2 \\
	2 & -1 & 8 \\	
\end{vmatrix}
\begin{matrix}
	3 & 4 \\
	8 & 7 \\
	2 & -1
\end{matrix}
= (3)(7)(8) + (4)(-2)(2) + (-5)(8)(-1) - (2)(7)(-5) - (-1)(-2)(3) - (8)(8)(4) =
168 - 16 + 40 + 70 - 6 - 256 = 0
$

\subsubsection{Способ 2. Разложение}

$
\begin{vmatrix}
	a & b & c \\
	d & e & f \\
	m & n & k	
\end{vmatrix}
= \sum_{i=1}^{3} a_{i2} A_{i2} =
bA_{12} + eA_{22} + nA_{32}
$

$
\begin{vmatrix}
	3 & 4 & -5 \\
	8 & 7 & -2 \\
	2 & -1 & 8
\end{vmatrix}
= \sum_{i=1}^{3} a_{i2} A_{i2} =
4A_{12} + 7A_{22} + (-1)A_{32} =
4(-M_{12}) + 7M_{22} + M_{32}
$

где

$
M_{12} =
\begin{vmatrix}
	8 & -2 \\
	2 & 8
\end{vmatrix}
= 64 + 4 = 68
$

$
M_{22} =
\begin{vmatrix}
	3 & -5 \\
	2 & 8
\end{vmatrix}
= 24 + 10 = 34
$

$
M_{32} =
\begin{vmatrix}
	3 & -5 \\
	8 & -2
\end{vmatrix}
= -6 + 40 = 34
$

$
4(-68) + 7(34) + 34 = (-8 + 7 + 1) * 34 = 0
$

\subsubsection{Контроль}

$
\begin{vmatrix}
	3 & 2 & 1 \\
	2 & 5 & 3 \\
	3 & 4 & 2
\end{vmatrix}
\begin{matrix}
	3 & 2 \\
	2 & 5 \\
	3 & 4
\end{matrix}
= (3)(5)(2) + (2)(3)(3) + (1)(2)(4)
- (3)(5)(1) - (4)(3)(3) - (2)(2)(2) =
30 + 18 + 8 - 15 - 36 - 8 = -3
$

№2.54(a)

$
\begin{vmatrix}
	2 & -3 & 4 & 1 \\
	4 & -2 & 3 & 2 \\
	a & b & c & d \\
	3 & -1 & 4 & 3
\end{vmatrix}
= a A_{31} + b A_{32} + c A_{33} + d A_{34}
$

$
A_{31} = (-1)^{3+1} M_{31} =
\begin{vmatrix}
	-3 & 4 & 1 \\
	-2 & 3 & 2 \\
	-1 & 4 & 3
\end{vmatrix}
\begin{matrix}
	-3 & 4 \\
	-2 & 3 \\
	-1 & 4
\end{matrix}
= (-3)(3)(3) + (4)(2)(-1) + (1)(-2)(4)
- (-1)(3)(1) - (4)(2)(-3) - (3)(-2)(4) =
-27 - 8 - 8 + 3 + 24 + 24 = 8
$

$
A_{32} = (-1)^{3+2} M_{32} =
-\begin{vmatrix}
	2 & 4 & 1 \\
	4 & 3 & 2 \\
	3 & 4 & 3
\end{vmatrix}
\begin{matrix}
	2 & 4 \\
	4 & 3 \\
	3 & 4
\end{matrix}
= -(2)(3)(3) - (4)(2)(3) - (1)(4)(4)
+ (3)(3)(1) + (4)(2)(2) + (3)(4)(4) =
-(18 + 24 + 16 - 9 - 16 - 48) = 15
$

$
A_{33} = (-1)^{3+3} M_{33} =
\begin{vmatrix}
	2 & -3 & 1 \\
	4 & -2 & 2 \\
	3 & -1 & 3
\end{vmatrix}
\begin{matrix}
	2 & -3 \\
	4 & -2 \\
	3 & -1
\end{matrix}
= (2)(-2)(3) + (-3)(2)(3) + (1)(4)(-1)
- (3)(-2)(1) - (-1)(2)(2) - (3)(4)(-3) = ...
$

\subsubsection{Способ 3. С упрощением}

$
\begin{vmatrix}
	2 & -1 & 1 & 0 \\
	0 & 1 & 2 & -1 \\
	3 & -1 & 2 & 3 \\
	3 & 1 & 6 & 1
\end{vmatrix}
\begin{matrix}
	I \\
	II + IV \\
	III - 3IV \\
	IV
\end{matrix}
=
\begin{vmatrix}
	2 & -1 & 1 & 0 \\
	3 & 2 & 8 & 0 \\
	-6 & -4 & -16 & 0 \\
	3 & 1 & 6 & 1
\end{vmatrix}
= 0 A_{14} + 0 A_{24} + 0 A_{34} + 1 A_{44} =
M_{44} =
\begin{vmatrix}
	2 & -1 & 1 \\
	3 & 2 & 8 \\
	-6 & -4 & -16
\end{vmatrix}
= -2 * \begin{vmatrix}
	2 & -1 & 1 \\
	3 & 2 & 8 \\
	3 & 2 & 8
\end{vmatrix}
= -2 * 0 = 0
$

\subsubsection{Контроль}

№2.56

$
\begin{vmatrix}
	2 & 3 & -3 & 4 \\
	2 & 1 & -1 & 2 \\
	6 & 2 & 1 & 0 \\
	2 & 3 & 0 & -5
\end{vmatrix}
\begin{matrix}
	I - 3II \\
	II \\
	III + II \\
	IV
\end{matrix}
= \begin{vmatrix}
	-4 & 0 & 0 & -2 \\
	2 & 1 & -1 & 2 \\
	8 & 3 & 0 & 2 \\
	2 & 3 & 0 & 5
\end{vmatrix}
= -A_{23} = M_{23} =
\newline
= \begin{vmatrix}
	-4 & 0 & -2 \\
	8 & 3 & 2 \\
	2 & 3 & -5
\end{vmatrix}
= \begin{vmatrix}
	-4 & 0 & -2 \\
	6 & 0 & 7 \\
	2 & 3 & -5
\end{vmatrix}
= -3
\begin{vmatrix}
	-4 & -2 \\
	6 & 7
\end{vmatrix}
= -3 * (-28) + 12 = -3 * (-16) = 48
$

\subsubsection{ДЗ}

№№
\begin{itemize}
\item{2.1}
\item{.50}
\item{.52}
\item{.54(б)}
\item{.57}
\item{61*}
\end{itemize}

\end{document}
