\documentclass{article}
\usepackage[14pt]{extsizes}
\usepackage[a4paper, margin=2.5cm]{geometry}
\usepackage{indentfirst}
\usepackage[T1, T2A]{fontenc}
\usepackage[russian,english]{babel}
\usepackage{amsmath}
\usepackage{amssymb}
\DeclareMathOperator*{\Rg}{Rg}
\DeclareMathOperator*{\Tr}{tr}

\begin{document}
\title{Линейная алгебра}
\author{Илья Ковалев}
\date{6 сентября 2024}
\maketitle

\section{Экзамен}

Билет --- 5 вопросов, 2 теория и 3 практика

1 вопрос = 1 балл

Письменный экзамен, длительность --- 90 минут

\section{Матрицы и их операции}

Матрицей $A$ порядка $m * n$ называют двумерную таблицу, состоящую из $m$ строк и $n$ столбцов.

Прямоугольная матрица --- $m * n$.

Квадратная матрица --- $n * n$.

Диагональная матрица --- $n * n$, где отличны от нуля только элементы главной диагонали.

\[
\begin{pmatrix}
	a & 0 & 0 & 0\\
	0 & b & 0 & 0\\
	0 & 0 & c & 0\\
	0 & 0 & 0 & d
\end{pmatrix}
\]

Скалярная --- диагональная, где все элементы диагонали равны.

\[
\begin{pmatrix}
	a & 0 & 0\\
	0 & a & 0\\
	0 & 0 & a
\end{pmatrix}
\]

Единицная --- скалярная, где все элементы диагонали $= 1$.

\[
\begin{pmatrix}
	1 & 0 & 0\\
	0 & 1 & 0\\
	0 & 0 & 1
\end{pmatrix}
\]

Нулевая --- все элементы $= 0$.

\[
\begin{pmatrix}
	0 & 0 \\
	0 & 0
\end{pmatrix}
\]

\section{Вычисление определителей}

$detA_{n*m}$

\subsection{Младшие порядки}
\[
n = 2
\]

\[
\begin{vmatrix}
	a & b \\
	c & d
\end{vmatrix}
= ad - bc
\]

\[
\begin{vmatrix}
	-5 & -6 \\
	-7 & -8
\end{vmatrix}
= (-5)(-8) - (6)(-7) = 40 + 42 = 82
\]

\subsubsection{Способ 1. По Саррюсу}
\[
n = 3
\]

\subsubsection*{№2.13}

\begin{gather*}
\begin{vmatrix}
	3 & 4 & -5 \\
	8 & 7 & -2 \\
	2 & -1 & 8 \\	
\end{vmatrix}
\begin{matrix}
	3 & 4 \\
	8 & 7 \\
	2 & -1
\end{matrix} = \\
= (3)(7)(8) + (4)(-2)(2) + (-5)(8)(-1) - \\
- (2)(7)(-5) - (-1)(-2)(3) - (8)(8)(4) = \\
168 - 16 + 40 + 70 - 6 - 256 = 0
\end{gather*}

\subsubsection{Способ 2. Разложение}

\[
\begin{vmatrix}
	a & b & c \\
	d & e & f \\
	m & n & k	
\end{vmatrix} =
\sum_{i=1}^{3} a_{i2} A_{i2} =
bA_{12} + eA_{22} + nA_{32}
\]

\begin{gather*}
	\begin{vmatrix}
		3 & 4 & -5 \\
		8 & 7 & -2 \\
		2 & -1 & 8
	\end{vmatrix} = \sum_{i=1}^{3} a_{i2} A_{i2} = \\
	= 4A_{12} + 7A_{22} + (-1)A_{32} =
	4(-M_{12}) + 7M_{22} + M_{32}		
\end{gather*}

где

\begin{align*}
	M_{12} &= 
	\begin{vmatrix}
		8 & -2 \\
		2 & 8
	\end{vmatrix}
	= 64 + 4 = 68 \\
	M_{22} &=
	\begin{vmatrix}
		3 & -5 \\
		2 & 8
	\end{vmatrix}
	= 24 + 10 = 34 \\
	M_{32} &=
	\begin{vmatrix}
		3 & -5 \\
		8 & -2
	\end{vmatrix}
	= -6 + 40 = 34 \\
\end{align*}

\[
	4(-68) + 7(34) + 34 = (-8 + 7 + 1) * 34 = 0
\]

\subsubsection{Контроль}

\begin{gather*}
	\begin{vmatrix}
		3 & 2 & 1 \\
		2 & 5 & 3 \\
		3 & 4 & 2
	\end{vmatrix}
	\begin{matrix}
		3 & 2 \\
		2 & 5 \\
		3 & 4
	\end{matrix} = \\
	= (3)(5)(2) + (2)(3)(3) + (1)(2)(4) - \\
	- (3)(5)(1) - (4)(3)(3) - (2)(2)(2) = \\
	= 30 + 18 + 8 - 15 - 36 - 8 = -3
\end{gather*}

\subsubsection*{№2.54(a)}

\[
	\begin{vmatrix}
	2 & -3 & 4 & 1 \\
	4 & -2 & 3 & 2 \\
	a & b & c & d \\
		3 & -1 & 4 & 3
	\end{vmatrix}
	= a A_{31} + b A_{32} + c A_{33} + d A_{34}
\]

\begin{gather*}
	A_{31} = (-1)^{3+1} M_{31} = \\
	= \begin{vmatrix}
		-3 & 4 & 1 \\
		-2 & 3 & 2 \\
		-1 & 4 & 3
	\end{vmatrix}
	\begin{matrix}
		-3 & 4 \\
		-2 & 3 \\
		-1 & 4
	\end{matrix} = \\
	= (-3)(3)(3) + (4)(2)(-1) + (1)(-2)(4) - \\
	- (-1)(3)(1) - (4)(2)(-3) - (3)(-2)(4) = \\
	= -27 - 8 - 8 + 3 + 24 + 24 = 8
\end{gather*}

\begin{gather*}
	A_{32} = (-1)^{3+2} M_{32} = \\
	= -\begin{vmatrix}
		2 & 4 & 1 \\
		4 & 3 & 2 \\
		3 & 4 & 3
	\end{vmatrix}
	\begin{matrix}
		2 & 4 \\
		4 & 3 \\
		3 & 4
	\end{matrix} = \\
	= -(2)(3)(3) - (4)(2)(3) - (1)(4)(4) + \\
	+ (3)(3)(1) + (4)(2)(2) + (3)(4)(4) = \\
	= -(18 + 24 + 16 - 9 - 16 - 48) = 15	
\end{gather*}

\begin{gather*}
	A_{33} = (-1)^{3+3} M_{33} = \\
	= \begin{vmatrix}
		2 & -3 & 1 \\
		4 & -2 & 2 \\
		3 & -1 & 3
	\end{vmatrix}
	\begin{matrix}
		2 & -3 \\
		4 & -2 \\
		3 & -1
	\end{matrix} = \\
	= (2)(-2)(3) + (-3)(2)(3) + (1)(4)(-1) - \\
	- (3)(-2)(1) - (-1)(2)(2) - (3)(4)(-3) = \dots
\end{gather*}

\subsubsection{Способ 3. С упрощением}

\begin{gather*}
	\begin{vmatrix}
		2 & -1 & 1 & 0 \\
		0 & 1 & 2 & -1 \\
		3 & -1 & 2 & 3 \\
		3 & 1 & 6 & 1
	\end{vmatrix}
	\begin{matrix}
		I \\
		II + IV \\
		III - 3IV \\
		IV
	\end{matrix} = \\
	= \begin{vmatrix}
		2 & -1 & 1 & 0 \\
		3 & 2 & 8 & 0 \\
		-6 & -4 & -16 & 0 \\
		3 & 1 & 6 & 1
	\end{vmatrix} = \\
	= 0 A_{14} + 0 A_{24} + 0 A_{34} + 1 A_{44} =
	M_{44} = \\
	= \begin{vmatrix}
		2 & -1 & 1 \\
		3 & 2 & 8 \\
		-6 & -4 & -16
	\end{vmatrix}
	= -2 * \begin{vmatrix}
		2 & -1 & 1 \\
		3 & 2 & 8 \\
		3 & 2 & 8
	\end{vmatrix}
	= -2 * 0 = 0
\end{gather*}

\subsubsection{Контроль}

\subsubsection*{№2.56}

\begin{gather*}
	\begin{vmatrix}
		2 & 3 & -3 & 4 \\
		2 & 1 & -1 & 2 \\
		6 & 2 & 1 & 0 \\
		2 & 3 & 0 & -5
	\end{vmatrix}
	\begin{matrix}
		I - 3II \\
		II \\
		III + II \\
		IV
	\end{matrix} = \\
	= \begin{vmatrix}
		-4 & 0 & 0 & -2 \\
		2 & 1 & -1 & 2 \\
		8 & 3 & 0 & 2 \\
		2 & 3 & 0 & 5
	\end{vmatrix} = \\
	= -A_{23} = M_{23} = \begin{vmatrix}
		-4 & 0 & -2 \\
		8 & 3 & 2 \\
		2 & 3 & -5
	\end{vmatrix} = \\
	= \begin{vmatrix}
		-4 & 0 & -2 \\
		6 & 0 & 7 \\
		2 & 3 & -5
	\end{vmatrix}
	= -3
	\begin{vmatrix}
		-4 & -2 \\
		6 & 7
	\end{vmatrix} = \\
	= -3 * (-28) + 12 = -3 * (-16) = 48
\end{gather*}

\subsection{Свойства определителей}

\begin{enumerate}
	\item Транспонирование --- строки и столбцы равноправны
	\item Упрощение
	\item Перестановка двух строк/столбцов меняет знак определителя, не меняет модуль
	\item Умножение
	\[
	\begin{vmatrix}
		1 & 2 & 3 \\
		4 & 5 & 6 \\
		4 & 8 & 12
	\end{vmatrix} =
	4 \begin{vmatrix}
		1 & 2 & 3 \\
		4 & 5 & 6 \\
		1 & 2 & 3
	\end{vmatrix}
	\]
	\item Сложение
	\[
	\begin{vmatrix}
		1 & 2 & 3 \\
		4 & 5 & 6 \\
		7 & 8 & 9
	\end{vmatrix} +
	\begin{vmatrix}
		1 & 2 & 3 \\
		-1 & 3 & -4 \\
		7 & 8 & 9
	\end{vmatrix} =
	4 \begin{vmatrix}
		1 & 2 & 3 \\
		3 & 8 & 2 \\
		7 & 8 & 9
	\end{vmatrix}
	\]
	\item Спец-свойство
	\[
	i \ne j, \sum_{k=1}^{n} a_{ik} A_{jk} = 0
	\]
	\item Произведение
	\[
	det A * det B = det A * B
	\]
	\item Треугольный определитель
	\[
	det U_{4*\times 4} =
	\begin{vmatrix}
		a_{11} & a_{12} & a_{13} & a_{14} \\
		0 & a_{22} & a_{23} & a_{24} \\
		0 & 0 & a_{33} & a_{34} \\
		0 & 0 & 0 & a_{44} \\
	\end{vmatrix} =
	a_{11} * a_{22} * a_{33} * a_{44}
	\]
\end{enumerate}

\pagebreak
\section{Обратная матрица}

Деление в алгебре:

$ax = b \stackrel{a \ne 0}{\Rightarrow} x = \frac{b}{a}$

\textbf{Обратное число} для $a \ne 0$ --- такое $a^{-1}$, что $a * a^{-1} = 1$

\subsection{Определение}

Обратная матрица $A^{-1}$ --- такая, что ее произведение и слева, и справа --- единичная матрица.

\begin{align*}
A A^{-1} &\triangleq E \\
A^{-1} A &\triangleq E
\end{align*}

\subsection{Свойства}

\begin{enumerate}
	\item \textbf{Порядок} --- $A_{n \times n} \Rightarrow A_{n \times n}^{-1}$
	\item \textbf{Единственность} --- $A : \exists A^{-1}, \exists ! A^{-1}$\\
	Доказательство от противного:\\
	Предположим, что $A : \exists A^{-1}_1 \ne A^{-1}_2$\\
	Тогда $A A_1^{-1} - A A_2^{-1} = E - E = 0$\\
	$A(A_1^{-1} - A_2^{-1}) = 0$\\
	$A_1^{-1} - A_2^{-2} = 0 \Rightarrow A_1^{-1} = A_2^{-1}$ --- противоречие. ЧТД.
	\item \textbf{Обратность определителя} --- $A : \exists A^{-1} \Rightarrow det A = \frac{1}{det A^{-1}}$\\
	$A A^{-1} = E$\\
	$det A A^{-1} = det E = 1$\\
	$det A = \frac{1}{det A^{-1}}$
	\item \textbf{Ненулевость определителя} --- $det A = 0 \Rightarrow \not \exists A^{-1}$\\
	Если $A : \exists A^{-1}$, то $det A \ne 0$
	\item $A : det A \ne 0 \Rightarrow A^{-1} = \frac{A^*}{det A}$\\
	$A^* = adj(A) = (A_{ij})_{i,j = T,i}^T$
\end{enumerate}

\textbf{Вырожденная матрица} --- $A, det A = 0$

\subsection{Алгоритм обращения матрицы}

\begin{enumerate}
	\item $A^{-1} = \frac{A_{ij}^T}{det A}$
	\item Вычислить все $A_{ij}$ для $a_{ij}$.
	\item Собрать все $A_{ij}$ в матрицу и транспонировать ее.
	\[
	A^* = \begin{pmatrix}
		A_{11} & A_{12} & \dots & A_{1n} \\
		A_{21} & A_{22} & \dots & A_{2n} \\
		\vdots & \vdots & & \vdots \\
		A_{m1} & A_{m2} & \dots & A_{mn}
	\end{pmatrix}^T =
	\begin{pmatrix}
		A_{11} & A_{21} & \dots & A_{m1} \\
		A_{12} & A_{22} & \dots & A_{m2} \\
		\vdots & \vdots & & \vdots \\
		A_{1n} & A_{2n} & \dots & A_{mn}
	\end{pmatrix}
	\]
	\item Разделить. $A^{-1} = \frac{A^*}{det A}$
	\item Проверить. $A A^{-1} = E$
\end{enumerate}

\subsubsection{Пример --- быстрое обращение матрицы $2 \times 2$}

$
n = 2 : A = \begin{pmatrix}
	a & b \\
	c & d
\end{pmatrix}
$
\begin{enumerate}
	\item Пусть $det A = ad - bc \ne 0$
	\item Найдем все алгебраические дополнения 
	\begin{gather*}
		A_{11} = M_{11} = d \\
		A_{12} = -M_{12} = -c \\
		A_{21} = -M_{21} = -b \\
		A_{22} = M_{22} = a \\
	\end{gather*}
	\item \[ A^* = \begin{pmatrix}
		d & -c \\
		-b & a
	\end{pmatrix}^T =
	\begin{pmatrix}
		d & -b \\
		-c & a
	\end{pmatrix} \]
	\item \[ A^{-1} = \frac{A^*}{det A} \]
\end{enumerate}

\subsection{Свойства операций обращения матриц}

Примечание: $\forall A \exists A^{-1}$

\begin{enumerate}
	\item $(AB)^{-1} = A^{-1} B^{-1}$
	\item $(\lambda B)^{-1} = \lambda^{-1} B^{-1}$
	\item $(A^{-1})^{-1} = A$
	\item $(A^{-1})^T = (A^T)^{-1} = A^{-T}$
\end{enumerate}

\subsection{Примеры:}

\subsubsection*{№2.106}

\begin{gather*}
	n = 2 \\
	A = \begin{pmatrix}
		1 & 2 \\
		3 & 4
	\end{pmatrix} \\
	A^{-1} = \frac{1}{4 - 6} \begin{pmatrix}
		4 & -2 \\
		-3 & 1
	\end{pmatrix} \\
	A^{-1} = \begin{pmatrix}
		-2 & 1 \\
		1.5 & -0.5
	\end{pmatrix}
\end{gather*}

Проверка:

\begin{gather*}
	A A^{-1} = \begin{pmatrix}
		(1)(-2) + (2)(1.5) & (-2)(2) + (1)(4) \\
		(3)(-2) + (4)(1.5) & (1.5)(2) + (4)(-0.5)
	\end{pmatrix} = \\
	= \begin{pmatrix}
		1 & 0 \\
		0 & 1
	\end{pmatrix} = E_{2 \times 2}
\end{gather*}

\subsubsection*{№2.109}

\begin{enumerate}
	\item Определитель
	\begin{gather*}
		n = 3 \\
		A = \begin{pmatrix}
			2 & 5 & 7 \\
			6 & 3 & 4 \\
			5 & -2 & -3
		\end{pmatrix} \\
		det A = \begin{vmatrix}
			2 & 5 & 7 \\
			6 & 3 & 4 \\
			5 & -2 & -3
		\end{vmatrix} = \dots = -1 \ne 0 \Rightarrow \exists A^{-1}
	\end{gather*}
	\item Первая строка
	\begin{gather*}
		A_{11} = M_{11} = \begin{vmatrix}
			3 & 4 \\
			-2 & -3
		\end{vmatrix} = -9 + 8 = -1 \\
		A_{12} = -M_{12} = -\begin{vmatrix}
			6 & 4 \\
			5 & -3
		\end{vmatrix} = -(-18 + (-20)) = 38 \\
		A_{13} = M_{13} = \begin{vmatrix}
			6 & 3 \\
			5 & -2
		\end{vmatrix} = -12 - 15 = -27
	\end{gather*}
	\item Вторая строка
	\begin{gather*}
		A_{21} = -M_{21} = \begin{vmatrix}
			5 & 7 \\
			-2 & -3
		\end{vmatrix} = -(-15 + 14) = 1 \\
		A_{22} = M_{22} = -\begin{vmatrix}
			2 & 7 \\
			5 & -3
		\end{vmatrix} = -6 + (-35) = -41 \\
		A_{23} = -M_{23} = \begin{vmatrix}
			2 & 5 \\
			5 & -2
		\end{vmatrix} = -(-4 - 25) = 29
	\end{gather*}
	\item Третья строка
	\begin{gather*}
		A_{31} = M_{31} = \begin{vmatrix}
			5 & 7 \\
			3 & 4
		\end{vmatrix} = 20 - 21 = -1 \\
		A_{32} = -M_{32} = -\begin{vmatrix}
			2 & 7 \\
			6 & 4
		\end{vmatrix} = -(8 - 42) = 34 \\
		A_{33} = M_{33} = \begin{vmatrix}
			2 & 5 \\
			6 & 3
		\end{vmatrix} = 6 - 30 = -24
	\end{gather*}
	\item Присоединенная матрица
	\begin{gather*}
		A^* = \begin{pmatrix}
			A_11 & A_21 & A_31 \\
			A_12 & A_22 & A_32 \\
			A_13 & A_23 & A_33
		\end{pmatrix} = \begin{pmatrix}
			-1 & 1 & -1 \\
			38 & -41 & 34 \\
			-27 & 29 & -24
		\end{pmatrix}
	\end{gather*}
	\item Обратная матрица
	\begin{gather*}
		A^{-1} = \frac{A^*}{-1} = \begin{pmatrix}
			1 & -1 & 1 \\
			-38 & 41 & -34 \\
			27 & -29 & 24
		\end{pmatrix}
	\end{gather*}
\end{enumerate}




\pagebreak
\section{Решение линейных уравнений}

\subsubsection*{Тип 1}

\begin{align*}
A_{m \times n} X_{n \times P} &= B_{m \times P}\\
A_{n \times m}^{-1} A_{m \times n} X_{n \times P} &= A_{n \times m}^{-1} B_{m \times P}\\
X_{n \times P} &= A_{n \times m}^{-1} B_{m \times P}
\end{align*}

\subsubsection*{Тип 2}

\begin{align*}
Y A &= B \\
Y A A^{-1} &= B A^{-1} \\
Y &= B A^{-1}
\end{align*}

\subsubsection*{Тип 3}

\begin{align*}
	A_1 X A_2 &= B \\
	A_1^{-1} A_1 X A_2 A_2^{-1} &= A_1^{-1} B A_2^{-1} \\
	Y &= A_1^{-1} B A_2^{-1}
\end{align*}	

\subsubsection*{Тип 4}

$A_1 X + X A_2 = B$ --- не вычисляется с помощью обратных матриц

\subsection{Значения матриц}

$
X = \vec{x} = \begin{pmatrix}
	x_1 \\
	x_2 \\
	\vdots \\
	x_n
\end{pmatrix}
$ --- переменные

$B = \vec{b} = \begin{pmatrix}
	b_1 \\
	b_2 \\
	\vdots \\
	b_m
\end{pmatrix}$ --- свободные члены

$A = \begin{pmatrix}
	a_{11} & a_{12} & \dots & a_{1n} \\
	a_{21} & a_{22} & \dots & a_{2n} \\
	\vdots & \vdots & & \vdots \\
	a_{m1} & a_{m2} & \dots & a_{mn} \\
\end{pmatrix}$ --- коэффициенты

\subsection{Упражнения}

\subsubsection*{№2.121}

\begin{enumerate}
	\item \[
	\begin{pmatrix}
		1 & 2 \\
		3 & 4
	\end{pmatrix} X =
	\begin{pmatrix}
		3 & 5 \\
		5 & 9
	\end{pmatrix}
	\]
	\item \begin{gather*}
		X = A^{-1} B = \\
		= -\frac{1}{2}
		\begin{pmatrix}
			4 & -2 \\
			-3 & 1
		\end{pmatrix}
		\begin{pmatrix}
			3 & 5 \\
			5 & 9
		\end{pmatrix} = \\
		= -\frac{1}{2}
		\begin{pmatrix}
			2 & 2 \\
			-4 & -6
		\end{pmatrix} = \\
		\begin{pmatrix}
			-1 & -1 \\
			2 & 3
		\end{pmatrix}
	\end{gather*}
	\item Проверка
	\begin{gather*}
		A X \stackrel{?}{=} B \\
		A X = \begin{pmatrix}
			1 & 2 \\
			3 & 4
		\end{pmatrix}
		\begin{pmatrix}
			-1 & -1 \\
			2 & 3
		\end{pmatrix}
		= \begin{pmatrix}
			3 & 5 \\
			5 & 9
		\end{pmatrix}
	\end{gather*}
	Верно
\end{enumerate}

\subsubsection*{№3.122}

\[
X \begin{pmatrix}
	3 & -2 \\
	5 & -4
\end{pmatrix} =
\begin{pmatrix}
	-1 & 2 \\
	-5 & 6
\end{pmatrix}
\]

\begin{enumerate}
	\item \[
	A^{-1} = \frac{1}{det A} \begin{pmatrix}
		-4 & 2 \\
		-5 & 3
	\end{pmatrix} =
	\frac{1}{-2} \begin{pmatrix}
		-4 & 2 \\
		-5 & 3
	\end{pmatrix}
	\]
	\item $X = \dots$ --- мне лень
\end{enumerate}

\subsection{Нахождение решений}

\begin{gather*}
	\Delta = det A \\
	m = n \land \Delta \ne 0 \Rightarrow x_j = \frac{\Delta_j}{\Delta} \\
	\Delta_j = D(a_1, \dots a_{j-1}, b, a_{j+1}, \dots a_n)
\end{gather*}

\subsubsection*{Пример:}

\begin{gather*}
	\begin{cases}
		2x_1 + 3x_2 = 4 \\
		5x_1 - 7x_2 = 6
	\end{cases} \Rightarrow
	\begin{pmatrix}
		2 & 3 \\
		5 & -7
	\end{pmatrix} \vec{x} =
	\begin{pmatrix}
		4 \\
		6
	\end{pmatrix} \\
	\Delta = -29 \ne 0 \\
	x_1 = \frac{\begin{vmatrix}
		4 & 3 \\
		6 & -7
	\end{vmatrix}}{\Delta} = \frac{46}{29} \\
	x_1 = \frac{\begin{vmatrix}
		2 & 4 \\
		5 & 6
	\end{vmatrix}}{\Delta} = \frac{8}{29}
\end{gather*}

\subsection{Линейная зависимость}

Елси $c_1 \vec{a}_1 + c_2 \vec{a}_2 + \dots + c_k \vec{a}_k = \vec{0}$ только при
$c_1 = c_2 = \dots = c_k = 0$, то уравнение Линейно НеЗависимо (ЛНЗ)

Иначе, при $c_1 = c_2 \ne 0 \Rightarrow c_1 \vec{a}_1 + c_2 \vec{a}_2 = \vec{0} \Rightarrow \vec{a}_1 = \frac{c_2}{c_1} \vec{a}_2$

\subsubsection{Свойства ЛЗ и ЛНЗ}

\begin{enumerate}
	\item При сужении ЛНЗ, система остается ЛНЗ
	\item При расширении ЛЗ, система остается ЛЗ
\end{enumerate}

\subsection{Необходимое и достаточное условие $det A = 0$}

\begin{gather*}
	det A_{m \times n} = 0 \Leftrightarrow \exists \text{ЛЗ} \\
	\vec{a}_n = \alpha_1 \vec{a}_k + \alpha_2 \vec{a}_l \Rightarrow
	\vec{a}_n - \alpha_1 \vec{a}_k - \alpha_2 \vec{a}_l = \vec{0} \Rightarrow det A = 0
\end{gather*}

\subsection{Минор прямоугольной матрицы}

Минором прямоугольной матрицы $A_{m \times n}$ порядка $k \le \min(m, n)$ \\
называется любой определитель $M_k$, составленный из элементов на \\
пересечении $k$ различных строк и столбцов.

\subsubsection*{Пример:}

\[
A_{3 \times 5} = \begin{pmatrix}
	1 & 2 & 3 & 4 & 5 \\
	6 & 7 & 8 & 9 & 10 \\
	11 & 12 & 13 & 14 & 15
\end{pmatrix}
\]

\begin{gather*}
	k = 1 \Rightarrow \\
	M_{1}^{'} = M_{II}^{II} = 7 \\
	M_{2}^{'} = M_{III}^{I} = 11 \\
\end{gather*}

\begin{gather*}
	k = 2 \Rightarrow \\
	M_{1}^{''} = M_{II, III}^{II, III} = \begin{vmatrix}
		7 & 8 \\
		12 & 13
	\end{vmatrix} \\
	M_{2}^{''} = M_{I, V}^{I, III} = \begin{vmatrix}
		1 & 5 \\
		11 & 15
	\end{vmatrix}
\end{gather*}

\subsection{Базис минора матрицы}

\[
M_{\text{баз}} = M_r \ne 0 : \forall M_{r+1} = 0\ \lor \not \exists
\]

\subsection{Ранг матрицы}

$\Rg{A}$ --- ранг матрицы $A$

\textbf{Ранг матрицы} --- количество ее линейно независимых строк \\
или столбцов

\subsubsection*{Свойство ранга}
\[ \Rg{A} = \dim M_{\text{баз}} \]

\subsection{Нахождение $\Rg{A}$ и $M_{\text{баз}}$}

\[
A_{3 \times 5} = \begin{pmatrix}
	1 & 2 & 3 & 4 & 5 \\
	6 & 7 & 8 & 9 & 10 \\
	11 & 12 & 13 & 14 & 15
\end{pmatrix}
\]

Пусть $r = 1$
\[
	M_1^{'} = 8 \ne 0 \rightarrow \Rg{A} \ge 1, M_{\text{баз}} = 8
\]

Пусть $r = 2$
\[
	M_2^{''} = \begin{vmatrix}
		2 & 3 \\
		7 & 8
	\end{vmatrix}
	= 5 \ne 0 \rightarrow \Rg{A} \ge 2, M_{\text{баз}} = 5
\]

Пусть $r = 3$
\[
	M_3^{'''} = \begin{vmatrix}
		1 & 2 & 3 \\
		6 & 7 & 8 \\
		11 & 12 & 13
	\end{vmatrix}
	= 5 \ne 0 \rightarrow \Rg{A} \ge 2, M_{\text{баз}} = 5
\]

\subsection{Операции}

\begin{enumerate}
	\item Перестановка строк/столбцов
	\item Умножение строки/столбца на $c \ne 0$
	\item Сложение строк/столбцов
\end{enumerate}

\subsection{Ступенчатая матрица}

\[
\begin{pmatrix}
	0 & * & * & \dots & * \\
	0 & 0 & * & \dots & 0 \\
	\vdots & \vdots & \vdots & \vdots & \vdots \\
	0 & 0 & 0 & \dots & * \\
	0 & 0 & 0 & \dots & 0 \\
\end{pmatrix}
\]

где $*$ --- не нуль

\textbf{Ступенчатая матрица} --- матрица, в которой можно провести \\
ломаную линию, отделяющую все ненулевые элементы от остальных, \\
при этом граничные элементы не могут быть нулями.

\subsection{Правило крамера}

\begin{gather*}
	n \times n \begin{cases}
		a_{11} x_1 + \dots + a_{1n} x_n = b_1 \\
		\vdots \\
		a_{n1} x_1 + \dots + a_{nn} x_n = b_n
	\end{cases}
	\Leftrightarrow
	A \vec{x} = \vec{b} \\
	A = \begin{pmatrix}
		a_{11} \dots a_{1n} \\
		\vdots \\
		a_{n1} \dots a_{nn}
	\end{pmatrix} \\
	\vec{x} = \begin{pmatrix}
		x_1 \\
		\vdots \\
		x_n
	\end{pmatrix} \\
	\vec{b} = \begin{pmatrix}
		b_1 \\
		\vdots \\
		b_n
	\end{pmatrix} \\
	\det A \ne 0 \Rightarrow \exists ! A^{-1} \Rightarrow \exists \vec{x} = A^{-1} \vec{b} \\
	x_j = \frac{\Delta j}{\Delta}, \Delta = \det A, \Delta_j = D(\vec{a}_1, \dots \vec{a}_{j - 1}, \vec{b}, \vec{a}_{j + 1}, \dots \vec{a}_n)
\end{gather*}

\subsubsection*{№2.191}

\begin{align*}
	\begin{cases}
		2x + y = 5 \\
		x + 3z = 16 \\
		5y - z = 10
	\end{cases}
	&\Rightarrow
	\vec{x} = \begin{pmatrix}x \\ y \\ z\end{pmatrix},
	\vec{b} = \begin{pmatrix}5 \\ 16 \\ 10\end{pmatrix} \\
	A &= \begin{pmatrix}
		2 & 1 & 0 \\
		1 & 0 & 3 \\
		0 & 5 & -1
	\end{pmatrix} \\
	\Delta &= -29 \ne 0 \\
	\Delta_1 &= \begin{vmatrix}
		5 & 1 & 0 \\
		16 & 0 & 3 \\
		10 & 5 & -1
	\end{vmatrix} = -29 \\
	\frac{\Delta_1}{\Delta} &= 1 \\
	\Delta_1 &= \begin{vmatrix}
		2 & 5 & 0 \\
		1 & 16 & 3 \\
		0 & 10 & -1
	\end{vmatrix} = -87 \\
	\frac{\Delta_2}{\Delta} &= 3 \\
	\Delta_1 &= \begin{vmatrix}
		2 & 1 & 5 \\
		1 & 0 & 16 \\
		0 & 5 & 10
	\end{vmatrix} = -145 \\
	\frac{\Delta_2}{\Delta} &= 5 \\
	\vec{x} &= \begin{pmatrix}1 \\ 3 \\ 5\end{pmatrix}
\end{align*}

\subsection{Метод окаймляющих миноров матрицы}

\[
A_{3 \times 5} = \begin{pmatrix}
	2 & -1 & 3 & -2 & 4 \\
	4 & -2 & 5 & 1 & 7 \\
	2 & -1 & 1 & 8 & 2 \\
\end{pmatrix}
\]

\begin{gather*}
	k = 1: M_{II}^{III} = 5 \ne 0 \Rightarrow r \ge 1 \\
	k = 2: M_{I, II}^{II, III} = \begin{vmatrix}
		-1 & 3 \\
		-2 & 5
	\end{vmatrix} = -5 + 6 = 1 \ne 0 \Rightarrow r \ge 2 \\
	k = 3: M' = M_{I, II, III}^{I, II, III} = \begin{vmatrix}
		2 & -1 & 3 \\
		4 & -2 & 5 \\
		2 & -1 & 1
	\end{vmatrix} = 0 \text{ --- не подходит на $M_\text{баз}$} \\
	M'' = M_{I, II, III}^{II, III, IV} = \begin{vmatrix}
		-1 & 3 & -2 \\
		-2 & 5 & 1 \\
		-1 & 1 & 8
	\end{vmatrix} = 0 \text{ --- не подходит на $M_\text{баз}$} \\
	M''' = M_{I, II, III}^{III, IV, V} = \begin{vmatrix}
		-1 & 3 & 4 \\
		-2 & 5 & 7 \\
		-1 & 1 & 2
	\end{vmatrix} = 0 \text{ --- не подходит на $M_\text{баз}$}
\end{gather*}

Итого:

\begin{gather*}
	M_\text{баз} = M_2 = M_{I, II}^{II, III} \\
	\Rg{A_{3 \times 5}} = 2
\end{gather*}

\subsubsection*{№2.151}

\[
A_{4 \times 4} = \begin{pmatrix}
	1 & 3 & 5 & -1 \\
	2 & -1 & -3 & 4 \\
	5 & 1 & -1 & 7 \\
	7 & 7 & 9 & 1
\end{pmatrix}
\]
\begin{gather*}
	k = 2 \\
	M_2 = M_{II, III}^{II, III} = \begin{vmatrix}
		-1 & -3 \\
		1 & -1
	\end{vmatrix} = 4 \ne 0 \Rightarrow r \ge 2 \\
\end{gather*}

\begin{gather*}
	k = 3 \\
	M'_3 = M_{I, II, III}^{I, II, III} = \begin{vmatrix}
		1 & 3 & 5 \\
		2 & -1 & -3 \\
		5 & 1 & -1
	\end{vmatrix} = \dots = 0 \\
	M''_3 = M_{I, II, III}^{II, III, IV} = \begin{vmatrix}
		2 & -1 & -3 \\
		5 & 1 & -1 \\
		7 & 7 & 9
	\end{vmatrix} = \dots = 0 \\
	M'_3 = M_{II, III, IV}^{II, III, IV} = \begin{vmatrix}
		-1 & -3 & 4 \\
		1 & -1 & 7 \\
		7 & 9 & 1
	\end{vmatrix} = -16 \ne 0 \Rightarrow r \ge 3
\end{gather*}
\begin{gather*}
	k = 4 \\
	M_4 = \det A = \begin{vmatrix}
		1 & 3 & 5 & -1 \\
		2 & -1 & -3 & 4 \\
		5 & 1 & -1 & 7 \\
		7 & 7 & 9 & 1
	\end{vmatrix} = 0 \land \exists ! M_4 \Rightarrow M_\text{баз} = M_3
\end{gather*}

\subsection{Метод элементарных преобразований матрицы. \\
Приведение к ступенчатому виду.}

\subsubsection{Правило}

Ранг ступенчатой матрицы равен числу ненулевых строк

\subsubsection*{№2.150}

\begin{gather*}
	A = \begin{pmatrix}
		2 & -1 & 3 & -2 & 4 \\
		4 & -2 & 5 & 1 & 7 \\
		2 & -1 & 1 & 8 & 2
	\end{pmatrix}
	\begin{bmatrix}
		\to 6 \\
		\to 15 \\
		\to 12
	\end{bmatrix}
	\begin{matrix}
		\\ II - 2I \\ II - I
	\end{matrix} \sim \\
	\sim \begin{pmatrix}
		2 & -1 & 3 & -2 & 4 \\
		0 & 0 & -1 & 5 & -1 \\
		0 & 0 & -2 & 10 & -2
	\end{pmatrix}
	\begin{bmatrix}
		\to 6 \\
		\to 3 \\
		\to 6
	\end{bmatrix} \sim \\
	\sim \begin{pmatrix}
		2 & -1 & 3 & -2 & 4 \\
		0 & 0 & -1 & 5 & -1 \\
		0 & 0 & 0 & 0 & 0
	\end{pmatrix}
	\begin{bmatrix}
		\to 6 \\
		\to 3 \\
		\to 0
	\end{bmatrix}
\end{gather*}

\[
\Rg{A_{3 \times 5}} = 2
\]

\subsection{Количество решений}

\begin{enumerate}
	\item Если количество неизвестных равно рангу матрицы, существует \\
	единственное решение
	\item Если количество неизвестных больше ранга матрицы, $\tilde{A}_{n \times r} \tilde{x}_{r+1}^{\text{баз}} =
	\tilde{b}_{r \times 1} = \tilde{A}_{n \times (r - 1)} \tilde{x}_{(r + 1) \times 1}^{\text{баз}}$
	\item Если количество неизвестных меньше ранга матрицы, то бесконечное \\
	количество решений, зависящих от $r - n$ параметров.
\end{enumerate}

\subsection{Метод Гаусса}

СЛАУ $4 \times 4$

\begin{gather*}
	\begin{cases}
		x_1 + 2x_2 + 3x_3 - x_4 = 1 \\
		x_1 - x_2 + x_3 + 2x_4 = 5 \\
		3x_1 + 5x_2 + 5x_3 - 4x_4 = -3 \\
		x_1 + 2x_2 + 7x_3 - 7x_4 = -7
	\end{cases} \\
	\begin{pmatrix}
		1 & 2 & 3 & -1 & \vline & 1 \\
		1 & -1 & 1 & 2 & \vline & 5 \\
		3 & 5 & 5 & -4 & \vline & -3 \\
		1 & 8 & 7 & -7 & \vline & -7
	\end{pmatrix} \sim \\
	\begin{pmatrix}
		1 & 2 & 3 & -1 & \vline & 1 \\
		0 & -1 & 4 & -1 & \vline & -6 \\
		0 & -3 & -2 & 3 & \vline & 4 \\
		0 & 6 & 4 & -6 & \vline & -8
	\end{pmatrix} \sim \\
	\begin{pmatrix}
		1 & 2 & 3 & -1 & \vline & 1 \\
		0 & -1 & 4 & -1 & \vline & -6 \\
		0 & 0 & -14 & 6 & \vline & 22 \\
		0 & 0 & 28 & -12 & \vline & -44
	\end{pmatrix} \sim \\
	\begin{pmatrix}
		1 & 2 & 3 & -1 & \vline & 1 \\
		0 & -1 & 4 & -1 & \vline & -6 \\
		0 & 0 & 0 & 0 & \vline & 0 \\
		0 & 0 & 0 & 0 & \vline & 0
	\end{pmatrix} \\
	\Rg{A} = 2 \\
	n - r = 4 - 2 = 2 \text{ --- число параметров}
\end{gather*}

\subsubsection{Алгоритм метода Гаусса}

\begin{enumerate}
	\item Привести расширенную матрицу $(A | b)$ к ступенчатому виду \\
	с помощью элементарных преобразований только строк матрицы.
	\item Определить ранги расширенной матрицы и матрицы коэффициентов \\
	и совместность системы.
	\item Разделить переменные на базисные колонны, выбрав подходящий \\
	базисный минор.
	\item Выразить базисные переменные через свободные. \\
	Если свободных переменных нет, одно решение.
	\item Записать ответ.
\end{enumerate}

\subsection{Однородные СЛАУ и их ФСР}

\begin{gather*}
	A\vec{x} = \vec{0} \Rightarrow \exists \vec{x} = \vec{0} \\
	r = \Rg{(A | \vec{0})}
\end{gather*}
Есть $\vec{x} \ne \vec{0}$

\subsubsection*{№2.226}

\begin{enumerate}
	\item Лестничный вид
	\begin{gather*}
		\begin{cases}
			2x_1 - 3x_2 + x_3 = 0 \\
			x_1 + x_2 + x_3 = 0 \\
			3x_1 - 2x_2 + 2x_3 = 0
		\end{cases} \\
		(A | \vec{b}) = \begin{pmatrix}
			1 & 1 & 1 & 0 \\
			2 & -3 & 1 & 0 \\
			3 & -2 & 2 & 0
		\end{pmatrix} \begin{matrix}
			I \\ II - 2I \\ III - 3I
		\end{matrix} \sim \\
		\sim \begin{pmatrix}
			1 & 1 & 1 & 0 \\
			0 & -5 & -1 & 0 \\
			0 & -5 & -1 & 0
		\end{pmatrix} \sim
		\begin{pmatrix}
			1 & 1 & 1 & 0 \\
			0 & -5 & -1 & 0 \\
			0 & 0 & 0 & 0
		\end{pmatrix} \\
		\Rg{A} = r = 2 < 3 = n
	\end{gather*}
	\item Обратный ход
	\begin{enumerate}
		\item Скалярный
		\[
			M_{\text{баз}} = \stackrel{\begin{matrix}
				x_1 & x_2
			\end{matrix}}{\begin{vmatrix}
				1 & 1 \\
				0 & -5 \\
			\end{vmatrix}} \ne 0
		\]
		$x_1, x_2$ --- базисные, $x_3$ --- свобод. \\
		\[
		\begin{cases}
			x_1 + x_2 = -x_3 \\
			-5x_2 = x_3
		\end{cases} \Rightarrow \begin{cases}
			x_2 = -0.2x_3 \\
			x_1 = -0.8x_3
		\end{cases}
		\]
		\item Матричный
		\begin{gather*}
			(A | \vec{b}) \sim \begin{pmatrix}
				1 & 1 & 1 & 0 \\
				0 & -5 & -1 & 0 \\
				0 & 0 & 0 & 0
			\end{pmatrix} \begin{matrix}
				 I + II \\ II \\ III
			\end{matrix} \sim
			\begin{pmatrix}
				1 & -4 & 0 & 0 \\
				0 & -5 & -1 & 0 \\
				0 & 0 & 0 & 0	
			\end{pmatrix} \Rightarrow \\
			\Rightarrow \begin{cases}
				x_1 - 4x_2 = 0 \\
				-5x_2 - x_3 = 0
			\end{cases} \Leftrightarrow \begin{cases}
				x_1 = 4x_2 \\
				x_3 = -5x_2
			\end{cases}
		\end{gather*}
	\end{enumerate}
	\item Подстановка
	\[
	\begin{cases}
		x_1 = -0.8c \\
		x_2 = -0.2c \\
		x_3 = c \in \mathbb{R}
	\end{cases}
	\]
	\item ФСР \\
	\begin{tabular}{c|c}
		Перем. & Знач. \\
		\hline
		$x_1$ & -0.8 \\
		$x_2$ & -0.2 \\
		$x_3$ & 1
	\end{tabular}
	\[
	\vec{\phi}_1 = \begin{pmatrix}
		-0.8 \\ -0.2 \\ 1
	\end{pmatrix} \Rightarrow
	\vec{x} = c \vec{\phi}_1
	\]
\end{enumerate}

\subsection{Связь однородной и неоднородной \\
соответствующих систем}

\begin{gather*}
	A \vec{x} = \vec{b} \ne \vec{0} \leftrightarrow A \vec{x} = \vec{0} \\
	\vec{x}_{\text{одн}} = c_1 \vec{\phi}_1 + \dots + c_{n-r} \vec{\phi}_{n-r} \\ 
\end{gather*}

\subsubsection*{№2.210}

\[
	\begin{cases}
		2x_1 + 7x_2 + 3x_3 + x_4 = 6 \\
		3x_1 + 5x_2 + 2x_3 + 2x_4 = 4 \\
		9x_1 + 4x_2 + x_3 + 7x_4 = 2
	\end{cases}
\]

\begin{enumerate}
	\item Лестничный вид
	\begin{gather*}
		(A | \vec{b}) = \begin{pmatrix}
			2 & 7 & 3 & 1 & 6 \\
			3 & 5 & 2 & 2 & 4 \\
			9 & 4 & 1 & 7 & 2
		\end{pmatrix} \begin{matrix}
			I \\ 2II - 3I \\ III - 3II
		\end{matrix} \sim \\
		\sim \begin{pmatrix}
			2 & 7 & 3 & 1 & 6 \\
			0 & -11 & -5 & 1 & -10 \\
			0 & -11 & -5 & 1 & -10
		\end{pmatrix} \begin{matrix}
			I \\ II \\ III - II
		\end{matrix} \sim \begin{pmatrix}
			2 & 7 & 3 & 1 & 6 \\
			0 & -11 & -5 & 1 & -10 \\
			0 & 0 & 0 & 0 & 0
		\end{pmatrix} \\
		\Rg{A} = r = 2 < 4 = n
	\end{gather*}
	\item Обратный ход
	\[
	M_{\text{баз}} = \stackrel{\begin{matrix}
		x_1 & x_4
	\end{matrix}}{\begin{vmatrix}
		2 & 1 \\
		0 & 1
	\end{vmatrix}}
	\]
	$x_1, x_4$ --- базисные, $x_2, x_3$ --- свободные \\
	\begin{gather*}
		\begin{pmatrix}
			2 & 7 & 3 & 1 & 6 \\
			0 & -11 & -5 & 1 & -10 \\
			0 & 0 & 0 & 0 & 0
		\end{pmatrix} \begin{matrix}
			I - II \\ II \\ III
		\end{matrix} \sim
		\begin{pmatrix}
			2 & 18 & 8 & 0 & 16 \\
			0 & -11 & -5 & 1 & -10 \\
			0 & 0 & 0 & 0 & 0
		\end{pmatrix} \Rightarrow \\
		\Rightarrow \begin{cases}
			2x_1 + 18x_2 + 8x_3 = 16 \\
			-11x_2 -5x_3 + x_4 = -10
		\end{cases} \Leftrightarrow \begin{cases}
			x_1 = 8 - 9x_2 - 4x_3 \\
			x_4 = -10 + 11x_2 + 5x_3
		\end{cases}
	\end{gather*}
	\item Общее решение
	\begin{gather*}
		\begin{cases}
			x_1 = 8 - 9c_1 - 4c_2 \\
			x_2 = c_1 \\
			x_3 = c_2 \\
			x_4 = -10 + 11c_1 + 5c_2
		\end{cases} \\
		\vec{x} = \vec{\phi}_0 + c_1 \vec{\phi}_1 + c_2 \vec{\phi}_2
	\end{gather*}
	\item Декомпозиция (разложение)
	\begin{gather*}
		\vec{x} = \begin{pmatrix}
			x_1 \\ x_2 \\ x_3 \\ x_4
		\end{pmatrix} =
		\begin{pmatrix}
			8 & -9c_1 & -4c_2 \\
			0 & +c_1 & 0 \\
			0 & 0 & +c_2 \\
			-10 & +11c_1 & +5c_2
		\end{pmatrix} = \\
		= \begin{pmatrix}
			8 \\ 0 \\ 0 \\ -10
		\end{pmatrix} +
		c_1 \begin{pmatrix}
			-9 \\ 1 \\ 0 \\ 11
		\end{pmatrix} +
		c_2 \begin{pmatrix}
			-4 \\ 0 \\ 1 \\ 5
		\end{pmatrix}
	\end{gather*}
	\item Проверка...
\end{enumerate}

\subsection{Собственные векторы и значения матриц}

$\vec{x} = \begin{pmatrix}x_1 \\ \vdots \\ x_n\end{pmatrix}$ --- собственный, если
$\vec{x} \ne \vec{0}, A\vec{x} = \lambda \vec{x}$

Тогда $\lambda$ --- собственное значение, пара $\{\lambda, \vec{x} \ne \vec{0} \}$

\begin{enumerate}
	\item \begin{gather*}
		\det(A - \lambda E) = 0 \\
		\phi (\lambda) = (-\lambda)^n + \Tr A \cdot (-\lambda)^{n-1} + \dots + \det A \\
		\Tr A = \sum_{i=1}^{n} A_{ii} \text{ --- след матрицы}
	\end{gather*}
	Тогда $\lambda_1, \lambda_2 \dots \lambda_n$ --- спектр $A$
	\item \begin{gather*}
		\forall \lambda_x : (A - \lambda E) \vec{x} = \vec{0} \stackrel{\text{Гаусс}}{\rightarrow}
		\vec{x}^{(k)} = c_1 \vec{\phi}_1^{(k)} + \dots + c_{n-r_k} \vec{\phi}_{n-r_k}^{(k)} \\
		r_k = \Rg B_k \\
		c_1^2 + \dots + c_{n-r_k}^2 \ne 0
	\end{gather*}
\end{enumerate}

\subsubsection*{№3.135}

\[
	A = \begin{pmatrix}
		0 & 1 & 0 \\
		-4 & 4 & 0 \\
		-2 & 1 & 2
	\end{pmatrix} \rightarrow \{ \lambda \in \mathbb{C}, \vec{x} \ne \vec{0} \}
\]

\begin{enumerate}
	\item Характерестическая матрица
	\[
		B(\lambda) \triangleq A - \lambda E = \begin{pmatrix}
			0 & 1 & 0 \\
			-4 & 4 & 0 \\
			-2 & 1 & 2 \\
		\end{pmatrix} - \lambda \begin{pmatrix}
			1 & 0 & 0 \\
			0 & 1 & 0 \\
			0 & 0 & 1
		\end{pmatrix}
	\]
	Характеристический многочлен
	\begin{gather*}
		\phi (\lambda) = \det B(\lambda) = \begin{vmatrix}
			-\lambda & 1 & 0 \\
			-4 & 4 - \lambda & 0 \\
			-2 & 1 & 2 - \lambda
		\end{vmatrix} = \\
		= (2 - \lambda)(\lambda^2 - 4\lambda + 8) = -\lambda^3 + 6\lambda^2 - 12\lambda + 8
	\end{gather*}
	Проверки: \begin{enumerate}
		\item Коэффициент при $(-\lambda)^{3-1} = \lambda^2 : \Tr A = 0 + 4 + 2 = 6$
		\item Свободный член $\det A = \begin{vmatrix}
			0 & 1 & 0 \\
			-4 & 4 & 0 \\
			-2 & 1 & 2
		\end{vmatrix}$ = 8
	\end{enumerate}
	\item Характеристическое уравнение
	\begin{gather*}
		-\lambda^3 + 6\lambda^2 - 12\lambda + 8 = 0 \\
		\lambda^3 - 6\lambda^2 + 12\lambda - 8 = 0
	\end{gather*}
	Если есть $\lambda_1 \in \mathbb{Z} = \{ \pm 1, \pm 2 \dots \pm 8 \}$,
	то он является делителем свободного члена $\det A$.
	\begin{enumerate}
		\item Подбор $\lambda_1$ среди делителей $\det A = b : \{\pm 1, \pm 2, \pm 4, \pm 8 \}$ \\
		$\lambda_1 = 2$, т.к. $2^3 - 6 \cdot 2^2 + 12 \cdot 2 - 8 = 0$
		\item Понижение степени уравнения
		\[
			\lambda^3 - 6\lambda^2 + 12\lambda - 8 = (\lambda - 2)(\lambda^2 - 4\lambda + 4) = (\lambda - 2)^3 = 0
		\]
		\item Собственные значения: $\lambda_1 = \lambda_2 = \lambda_3 = 2$
	\end{enumerate}
	\item Нахождение ненулевого столбца для $\lambda_1 = 2 : (A - \lambda_1 E)\vec{x} = \vec{0}$
	Метод Гаусса: \begin{enumerate}
		\item Ступенчатый вид
		\begin{gather*}
			(B(2) | \vec{0}) = \begin{pmatrix}
				-2 & 1 & 0 & 0 \\
				-4 & 2 & 0 & 0 \\
				-2 & 1 & 0 & 0 \\
			\end{pmatrix} \sim \begin{pmatrix}
				-2 & 1 & 0 & 0 \\
				0 & 0 & 0 & 0 \\
				0 & 0 & 0 & 0 \\
			\end{pmatrix} \\
			\Rg B(2) = 1 \\
			M_{\text{баз}} = 1
		\end{gather*}
		\item Обратный ход
		\[
			\begin{cases}
				x_2 = 2x_1 \\
				0 = 0 \\
				0 = 0
			\end{cases}
		\]
		\item Замена свободных переменных
		\[	
			\begin{cases}
				x_1 = c_1 \\
				x_2 = 2c_1 \\
				x_3 = c_2
			\end{cases}
		\]
		\item Декомпозиция
		\[
			\vec{x} = \begin{pmatrix} x_1 \\ x_2 \\ x_3 \end{pmatrix} =
			c_1 \begin{pmatrix} 1 \\ 2 \\ 0 \end{pmatrix} +
			c_2 \begin{pmatrix} 0 \\ 0 \\ 1 \end{pmatrix},
			c_1^2 + c_2^2 \ne 0
		\]
		\item Проверка
		\begin{enumerate}
			\item \begin{gather*}
				\{ \lambda = 2, \vec{\phi}_1 \} \\
				\lambda_1 \vec{\phi}_1 = 2\phi_1 = 2 \begin{pmatrix}
					1 \\ 2 \\ 0
				\end{pmatrix} = \begin{pmatrix}
					2 \\ 4 \\ 0
				\end{pmatrix} \\
				A \vec{\phi}_1 = \dots = \begin{pmatrix}
					2 \\ 4 \\ 0
				\end{pmatrix} \\
				\lambda_1 \vec{\phi}_1 = A \vec{\phi}_1 
			\end{gather*}
			\item $\{ \lambda = 2, \vec{\phi}_2 \}$ аналогично
		\end{enumerate}
	\end{enumerate}
\end{enumerate}

\subsubsection{Упрощенный вариант}

\begin{gather*}
	A = \begin{pmatrix}
		2 & -1 & 2 \\
		5 & -3 & 3 \\
		-1 & 0 & -2
	\end{pmatrix},
	\lambda \in \{ -1, 3 \}
\end{gather*}

\begin{enumerate}
	\item Подбор $\det (A - \lambda E) = B(\lambda) = 0$
	\begin{gather*}
		B(\lambda) = \begin{vmatrix}
			2 - \lambda & -1 & 2 \\
			5 & -3 - \lambda & 3 \\
			-1 & 0 & -2 - \lambda
		\end{vmatrix} \\
		B(-1) = \begin{vmatrix}
			3 & -1 & 2 \\
			5 & -2 & 3 \\
			-1 & 0 & -1
		\end{vmatrix} = 0 \Rightarrow \lambda = -1 \\
		B(3) = \begin{vmatrix}
			2 - 3 & -1 & 2 \\
			5 & -3 - 3 & 3 \\
			-1 & 0 & 2 - 3
		\end{vmatrix} = -64 \ne 0 \Rightarrow \lambda \ne 3
	\end{gather*}
	\item Гаусс
	\begin{enumerate}
		\item Ступенчатый вид
		\begin{gather*}
			(B(-1) | \vec{0}) = \begin{pmatrix}
				3 & -1 & 2 & 0 \\
				5 & -2 & 3 & 0 \\
				-1 & 0 & -1 & 0
			\end{pmatrix} \sim \\
			\sim \begin{pmatrix}
				1 & 0 & 1 & 0 \\
				0 & -1 & -1 & 0 \\
				0 & -2 & -2 & 0
			\end{pmatrix} \sim \begin{pmatrix}
				1 & 0 & 1 & 0 \\
				0 & 1 & 1 & 0 \\
				0 & 0 & 0 & 0
			\end{pmatrix} 
		\end{gather*}
		\item Обратный ход
		\begin{gather*}
			\begin{cases}
				x_1 + x_3 = 0 \\
				x_2 + x_3 = 0 \\
			\end{cases} \Rightarrow \begin{cases}
				x_1 = -x_3 \\
				x_2 = -x_3
			\end{cases}
		\end{gather*}
		\item Подстановка
		\begin{gather*}
			\begin{cases}
				x_1 = -c \\
				x_2 = -c \\
				x_3 = c
			\end{cases}
		\end{gather*}
		\item Декомпозиция
		\begin{gather*}
			\vec{x} = c_1 \begin{pmatrix}
				-1 \\ -1 \\ 1
			\end{pmatrix}, c_1 \ne 0 \\
			\vec{\phi}_1 = \begin{pmatrix}
				-1 \\ -1 \\ 1
			\end{pmatrix}
		\end{gather*}
		\item Проверка \dots
	\end{enumerate}
\end{enumerate}

\end{document}
