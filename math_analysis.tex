\documentclass{article}
\usepackage[14pt]{extsizes}
\usepackage[a4paper, margin=2.5cm]{geometry}
\usepackage{indentfirst}
\usepackage[T1, T2A]{fontenc}
\usepackage[russian,english]{babel}
\usepackage{amsmath}
\usepackage{amssymb}

\begin{document}

\title{Математический Анализ}
\author{Илья Ковалев}
\date{2024 год}
\maketitle



\section{Учебники}

\begin{itemize}
\item Зорич Владимир Антонович --- математический анализ
\item Фихтенгольц --- Основы математического анализа
\item Димедович --- Сборник задач по математическому анализу
\end{itemize}

\section{Элементы теории множеств}


Множество --- набор элементов.

Пустое множество --- $\emptyset$.

Универсальное множество --- $U$ --- элементов рассматриевомого типа.

\section{Операции над множествами}

\subsection{Приадлежность}

$x \in A$ --- $x$ принадлежит $A$

\subsection{Подмножество}

$A \subset B$ если
$\forall x \in A \Rightarrow x \in B$

\subsection{Пересечение}

$A \cap B = \{x: x \in A \land x \in B\}$

\subsection{Объединение}

$A \cup B = \{x: x \in A \lor x \in B\}$

\subsection{Разность}

$A \setminus B = \{x: x \in A \land x \notin B\}$

\subsection{Дополнение}

$\overline{A} = U \setminus A = \{x: x \notin A\}$

\section{Логические высказывания}
Логическое высказывание --- повествовательное предлжение, \\
про которое можно сказать, истинно оно или ложно.

Предикат --- утверждение, зависящие от переменной (переменных),
превращающаяся в логическое высказывание при подстановке вместо переменной (переменных) ее значения.

Область истинности предиката --- множество значений переменной (переменных), \\
при которых этот предикат превращается в истинное высказывание.

\section{Операции над лог. высказываниями}

\subsection{Отрицание}
$
\begin{tabular}{c|c}
	A & $\overline{A}$ \\
	\hline
	0 & 1 \\
	1 & 0
\end{tabular}
$

\section{Область существования и \\
определения функции}

\subsection{}

$y = \sqrt{1 + x}$

$1 + x \ge 0$

$x \ge -1$

$x \in [-1; +\infty)$

\subsection{}

$y = \sqrt{2 + x - x^2}$

$2 + x - x^2 \ge 0$

$x^2 - x - 2 \le 0$

$(x - 2)(x + 1) \le 0$

$x \in (-\infty; -1] \cup [2; +\infty)$

\subsection{}

$y = \lg{\frac{x^2 - 3x + 2}{x + 1}}$

$\frac{x^2 - 3x + 2}{x + 1} > 0$

$\frac{(x - 2)(x - 1)}{x + 1} > 0$

$x \in (-1; 1) \cup (2; +\infty)$

\subsection{}

$y = \sqrt{sin2x}$

$sin2x \ge 0$

$x \in [\pi k; \frac{\pi}{2} + \pi k: k \in \mathbb{Z}]$

\section{Четность и нечетность}

\subsection{}

$f(x) = \frac{a^x + a^{-x}}{2}$

$f(-x) = \frac{a^{-x} + a^x}{2}$

$f(x) = f(-x)$ --- функция четная

\subsection{}

$f(x) = \sqrt[3]{(x - 1)^2} + \sqrt[3]{(x + 1)^2}$

$f(-x) = \sqrt[3]{(x + 1)^2} + \sqrt[3]{(x - 1)^2}$

$f(x) = f(-x)$ --- функция четная

\subsection{}

$f(x) = \lg{\frac{1 + x}{1 - x}}$

$f(-x) = \lg{\frac{1 - x}{1 + x}} = -\lg{\frac{1 + x}{1 - x}}$

$f(x) = -f(-x)$ --- функция нечетная

\section{Периодичность}

Период --- $T$

\subsection{}

$f(x) = 10\sin{3x}$

$\sin{\alpha}$ --- $T = 2\pi$

$f(x)$ --- $T = \frac{2\pi}{3}$

\subsection{}

$f(x) = \alpha \sin{\lambda x} + \beta \cos{\lambda x}$

$T = \frac{2\pi}{\lambda}$

\section{Графики}

\subsection{Парабола}

$y = a(x - x_0)^2 + y_0$

$y = x^2 - x + 2$

$y = (x - \frac{1}{2})$

\subsection{Кубическая парабола}

$y = a(x - x_0)^3 + y_0$

\subsection{Гипербола}

$y = \frac{a}{x - x_0} + y_0$

\subsection{ДЗ}

Демидович:
№153, 154, 157, 165, 254, 255

\pagebreak
\section{Бинарные отношения}

\subsection{Отношение эквивалентности}

\begin{enumerate}
	\item Рефлексивно
	\item Симметрично
	\item Транзитивно
\end{enumerate}

\subsection{Отношение частичного порядка}

\begin{enumerate}
	\item Рефлексивно
	\item Антисимметрично
	\item Транзитивно
\end{enumerate}

\subsection{Отношение линейного порядка}

\begin{enumerate}
	\item Антирефлексивно
	\item Антисимметрично
	\item Транзитивно
	\item Задано для каждой пары элементов можества
\end{enumerate}

\pagebreak
\section{Аксиоматика действительных чисел}

Действительными числами называется множество $\mathbb{R}$, над \\
элементами которого можно совершать операции \\
сложения и умножения, между которыми установлено отношение \\
линейного порядка, для которых выполнено свойство полноты, \\
подчиняющимися следующему набору аксиом:

\subsection{Аксиомы сложения}

Сложение: $a + b = c \in \mathbb{R}$

\begin{enumerate}
	\item \textbf{Существование нуля}: $\exists 0, \forall a \in \mathbb{R} : a + 0 = 0 + a = a$
	\item \textbf{Существование противоположного элемента}: \\
	$\forall a \in \mathbb{R}, \exists {-a} : a + (-a) = 0$
	\item \textbf{Ассоциативность}: $a + (b + c) = (a + b) + c$
	\item \textbf{Коммутативность}: $a + b = b + a$
\end{enumerate}

\textbf{Группа} --- объект, удовлетворяющий аксиомам 1-3

\textbf{Абелева группа} --- объект, удовлетворяющий аксиомам 1-4

\subsection{Аксиомы умножения}

Умножение: $a * b = c \in \mathbb{R}$

\begin{enumerate}
	\item \textbf{Существование единицы}: $\exists 1, \forall a \in \mathbb{R} : 1 * a = a * 1 = a$
	\item \textbf{Существование обратного элемента}: \\
	$\forall a \ne 0 \in \mathbb{R}, \exists a^{-1} : a * a^{-1} = 1$
	\item \textbf{Ассоциотивность}: $a * (b * c) = (a * b) * c$
	\item \textbf{Коммутативность}: $a * b = b * a$
\end{enumerate}

$\mathbb{R} \setminus \{0\}$ --- абелева группа по умножению

\textbf{Поле} --- алгебраический объект с 2 бинарными операциями, \\
подчиняющийся 8 аксиомам.

\subsection{Аксиома связи сложения и умножения}

\[
\forall a, b, c \in \mathbb{R} : (a + b) * c = a * c + b * c
\]

\subsection{Отношение порядка}

На $\mathbb{R}$ существует отношение $\ge$, подчиняющееся следущим аксиомам:

\begin{enumerate}
	\item \textbf{Рефлексивность}: $a \ge a$
	\item \textbf{Антисимметричность}: $a \ge b \land b \ge a \Rightarrow a = b$
	\item \textbf{Транзитивность}: $a \ge b \land b \ge c \Rightarrow a \ge c$
	\item \textbf{Аксиома, определяющая что порядок --- линейный}: \\
	$\forall a, b \in \mathbb{R} : a \ge b \lor b \ge a$
\end{enumerate}

{\large Замечание:}

\[ a \ge b \Rightarrow b \ge a \]
\[ a \ge b \land a \ne b \Rightarrow a > b \]

\subsection{Аксиома связи сложения и порядка}

\[ a \ge b \Rightarrow a + c \ge b + c \]

\subsection{Аксиома связи умножения и порядка}

\[ a \ge 0 \land b \ge 0 \Rightarrow a * b \ge 0 \]

\subsection{Аксиома полноты}

Пусть $U, W$ --- такие множества, что $\forall x \in U, y \in W : x \le y$

Тогда $\exists c \in \mathbb{R} : x \le c \le y$

\end{document}