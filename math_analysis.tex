\documentclass{article}
\usepackage[14pt]{extsizes}
\usepackage[a4paper, margin=2.5cm]{geometry}
\usepackage{indentfirst}
\usepackage[T1, T2A]{fontenc}
\usepackage[russian,english]{babel}
\usepackage{amsmath}
\usepackage{amssymb}

\begin{document}

\title{Математический Анализ}
\author{Илья Ковалев}
\date{2024 год}
\maketitle



\section{Учебники}

\begin{itemize}
\item Зорич Владимир Антонович --- математический анализ
\item Фихтенгольц --- Основы математического анализа
\item Димедович --- Сборник задач по математическому анализу
\end{itemize}

\section{Элементы теории множеств}


Множество --- набор элементов.

Пустое множество --- $\emptyset$.

Универсальное множество --- $U$ --- элементов рассматриевомого типа.

\section{Операции над множествами}

\subsection{Приадлежность}

$x \in A$ --- $x$ принадлежит $A$

\subsection{Подмножество}

$A \subset B$ если
$\forall x \in A \Rightarrow x \in B$

\subsection{Пересечение}

$A \cap B = \{x: x \in A \land x \in B\}$

\subsection{Объединение}

$A \cup B = \{x: x \in A \lor x \in B\}$

\subsection{Разность}

$A \setminus B = \{x: x \in A \land x \notin B\}$

\subsection{Дополнение}

$\overline{A} = U \setminus A = \{x: x \notin A\}$

\section{Логические высказывания}
Логическое высказывание --- повествовательное предлжение, про которое можно сказать, истинно оно или ложно.

Предикат --- утверждение, зависящие от переменной (переменных),
превращающаяся в логическое высказывание при подстановке вместо переменной (переменных) ее значения.

Область истинности предиката --- множество значений переменной (переменных), при которых этот предикат превращается в истинное высказывание.

\section{Операции над логическими высказываниями}

\subsection{Отрицание}
$
\begin{tabular}{c|c}
	A & $\overline{A}$ \\
	\hline
	0 & 1 \\
	1 & 0
\end{tabular}
$

% \subsection{Конъюнкция}
% $
% \begin{tabular}{c|c||c}
% 	A & B & A \cap B \\
% 	\hline
% 	0 & 0 & 0
% \end{tabular}
% $

\section{Область существования и определения функции}

\subsection{}

$y = \sqrt{1 + x}$

$1 + x \ge 0$

$x \ge -1$

$x \in [-1; +\infty)$

\subsection{}

$y = \sqrt{2 + x - x^2}$

$2 + x - x^2 \ge 0$

$x^2 - x - 2 \le 0$

$(x - 2)(x + 1) \le 0$

$x \in (-\infty; -1] \cup [2; +\infty)$

\subsection{}

$y = \lg{\frac{x^2 - 3x + 2}{x + 1}}$

$\frac{x^2 - 3x + 2}{x + 1} > 0$

$\frac{(x - 2)(x - 1)}{x + 1} > 0$

$x \in (-1; 1) \cup (2; +\infty)$

\subsection{}

$y = \sqrt{sin2x}$

$sin2x \ge 0$

$x \in [\pi k; \frac{\pi}{2} + \pi k: k \in \mathbb{Z}]$

\section{Четность и нечетность}

\subsection{}

$f(x) = \frac{a^x + a^{-x}}{2}$

$f(-x) = \frac{a^{-x} + a^x}{2}$

$f(x) = f(-x)$ --- функция четная

\subsection{}

$f(x) = \sqrt[3]{(x - 1)^2} + \sqrt[3]{(x + 1)^2}$

$f(-x) = \sqrt[3]{(x + 1)^2} + \sqrt[3]{(x - 1)^2}$

$f(x) = f(-x)$ --- функция четная

\subsection{}

$f(x) = \lg{\frac{1 + x}{1 - x}}$

$f(-x) = \lg{\frac{1 - x}{1 + x}} = -\lg{\frac{1 + x}{1 - x}}$

$f(x) = -f(-x)$ --- функция нечетная

\section{Периодичность}

Период --- $T$

\subsection{}

$f(x) = 10\sin{3x}$

$\sin{\alpha}$ --- $T = 2\pi$

$f(x)$ --- $T = \frac{2\pi}{3}$

\subsection{}

$f(x) = \alpha \sin{\lambda x} + \beta \cos{\lambda x}$

$T = \frac{2\pi}{\lambda}$

\section{Графики}

\subsection{Парабола}

$y = a(x - x_0)^2 + y_0$

$y = x^2 - x + 2$

$y = (x - \frac{1}{2})$

\subsection{Кубическая парабола}

$y = a(x - x_0)^3 + y_0$

\subsection{Гипербола}

$y = \frac{a}{x - x_0} + y_0$

\subsection{ДЗ}

Демидович:
№153, 154, 157, 165, 254, 255

\end{document}