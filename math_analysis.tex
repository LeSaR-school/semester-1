\documentclass{article}
\usepackage[14pt]{extsizes}
\usepackage[a4paper, margin=2.5cm]{geometry}
\usepackage{indentfirst}
\usepackage[T1, T2A]{fontenc}
\usepackage[russian,english]{babel}
\usepackage{amsmath}
\usepackage{amssymb}

\begin{document}

\title{Математический Анализ}
\author{Илья Ковалев}
\date{2024 год}
\maketitle



\section{Учебники}

\begin{itemize}
\item Зорич Владимир Антонович --- математический анализ
\item Фихтенгольц --- Основы математического анализа
\item Димедович --- Сборник задач по математическому анализу
\end{itemize}

\section{Элементы теории множеств}


Множество --- набор элементов.

Пустое множество --- $\emptyset$.

Универсальное множество --- $U$ --- элементов рассматриевомого типа.

\section{Операции над множествами}

\subsection{Принадлежность}

$x \in A$ --- $x$ принадлежит $A$

\subsection{Подмножество}

$A \subset B$ если
$\forall x \in A \Rightarrow x \in B$

\subsection{Пересечение}

$A \cap B = \{x: x \in A \land x \in B\}$

\subsection{Объединение}

$A \cup B = \{x: x \in A \lor x \in B\}$

\subsection{Разность}

$A \setminus B = \{x: x \in A \land x \notin B\}$

\subsection{Дополнение}

$\overline{A} = U \setminus A = \{x: x \notin A\}$

\section{Логические высказывания}
Логическое высказывание --- повествовательное предлжение, \\
про которое можно сказать, истинно оно или ложно.

Предикат --- утверждение, зависящие от переменной (переменных),
превращающаяся в логическое высказывание при подстановке вместо переменной (переменных) ее значения.

Область истинности предиката --- множество значений переменной (переменных), \\
при которых этот предикат превращается в истинное высказывание.

\section{Операции над лог. высказываниями}

\subsection{Отрицание}
\begin{tabular}{c|c}
	A & $\overline{A}$ \\
	\hline
	0 & 1 \\
	1 & 0
\end{tabular}

\section{Область существования и \\
определения функции}

\subsection{}

\begin{gather*}
	y = \sqrt{1 + x} \\
	1 + x \ge 0 \\
	x \ge -1 \\
	x \in [-1; +\infty)
\end{gather*}

\subsection{}

\begin{gather*}
	y = \sqrt{2 + x - x^2} \\
	2 + x - x^2 \ge 0 \\
	x^2 - x - 2 \le 0 \\
	(x - 2)(x + 1) \le 0 \\
	x \in (-\infty; -1] \cup [2; +\infty)	
\end{gather*}

\subsection{}

\begin{gather*}
	y = \lg{\frac{x^2 - 3x + 2}{x + 1}} \\
	\frac{x^2 - 3x + 2}{x + 1} > 0 \\
	\frac{(x - 2)(x - 1)}{x + 1} > 0 \\
	x \in (-1; 1) \cup (2; +\infty)
\end{gather*}

\subsection{}

\begin{gather*}
	y = \sqrt{sin 2x} \\
	sin 2x \ge 0 \\
	x \in [\pi k; \frac{\pi}{2} + \pi k: k \in \mathbb{Z}]
\end{gather*}

\section{Четность и нечетность}

\subsection{}

\begin{gather*}
	f(x) = \frac{a^x + a^{-x}}{2} \\
	f(-x) = \frac{a^{-x} + a^x}{2} \\
	f(x) = f(-x)
\end{gather*}
функция четная

\subsection{}

\begin{gather*}
	f(x) = \sqrt[3]{(x - 1)^2} + \sqrt[3]{(x + 1)^2} \\
	f(-x) = \sqrt[3]{(x + 1)^2} + \sqrt[3]{(x - 1)^2} \\
	f(x) = f(-x)
\end{gather*}
функция четная

\subsection{}

\begin{gather*}
	f(x) = \lg{\frac{1 + x}{1 - x}} \\
	f(-x) = \lg{\frac{1 - x}{1 + x}} = -\lg{\frac{1 + x}{1 - x}} \\
	f(x) = -f(-x)
\end{gather*}
функция нечетная

\section{Периодичность}

Период --- $T$

\subsection{}

\begin{gather*}
	f(x) = 10\sin{3x} \\
	\sin{\alpha}: T = 2\pi \\
	f(x): T = \frac{2\pi}{3}
\end{gather*}

\subsection{}

\begin{gather*}
	f(x) = \alpha \sin{\lambda x} + \beta \cos{\lambda x} \\
	T = \frac{2\pi}{\lambda}
\end{gather*} 


\section{Графики}

\subsection{Парабола}

\[ y = a(x - x_0)^2 + y_0 \]

\subsection{Кубическая парабола}

\[ y = a(x - x_0)^3 + y_0 \]

\subsection{Гипербола}

\[ y = \frac{a}{x - x_0} + y_0 \]

\pagebreak
\section{Бинарные отношения}

\subsection{Отношение эквивалентности}

\begin{enumerate}
	\item Рефлексивно
	\item Симметрично
	\item Транзитивно
\end{enumerate}

\subsection{Отношение частичного порядка}

\begin{enumerate}
	\item Рефлексивно
	\item Антисимметрично
	\item Транзитивно
\end{enumerate}

\subsection{Отношение линейного порядка}

\begin{enumerate}
	\item Антирефлексивно
	\item Антисимметрично
	\item Транзитивно
	\item Задано для каждой пары элементов можества
\end{enumerate}

\pagebreak
\section{Аксиоматика действительных чисел}

Действительными числами называется множество $\mathbb{R}$, над \\
элементами которого можно совершать операции \\
сложения и умножения, между которыми установлено отношение \\
линейного порядка, для которых выполнено свойство полноты, \\
подчиняющимися следующему набору аксиом:

\subsection{Аксиомы сложения}

Сложение: $a + b = c \in \mathbb{R}$

\begin{enumerate}
	\item \textbf{Существование нуля}: $\exists 0, \forall a \in \mathbb{R} : a + 0 = 0 + a = a$
	\item \textbf{Существование противоположного элемента}: \\
	$\forall a \in \mathbb{R}, \exists {-a} : a + (-a) = 0$
	\item \textbf{Ассоциативность}: $a + (b + c) = (a + b) + c$
	\item \textbf{Коммутативность}: $a + b = b + a$
\end{enumerate}

\textbf{Группа} --- объект, удовлетворяющий аксиомам 1-3

\textbf{Абелева группа} --- объект, удовлетворяющий аксиомам 1-4

\subsection{Аксиомы умножения}

Умножение: $a * b = c \in \mathbb{R}$

\begin{enumerate}
	\item \textbf{Существование единицы}: $\exists 1, \forall a \in \mathbb{R} : 1 * a = a * 1 = a$
	\item \textbf{Существование обратного элемента}: \\
	$\forall a \ne 0 \in \mathbb{R}, \exists a^{-1} : a * a^{-1} = 1$
	\item \textbf{Ассоциотивность}: $a * (b * c) = (a * b) * c$
	\item \textbf{Коммутативность}: $a * b = b * a$
\end{enumerate}

$\mathbb{R} \setminus \{0\}$ --- абелева группа по умножению

\textbf{Поле} --- алгебраический объект с 2 бинарными операциями, \\
подчиняющийся 8 аксиомам.

\subsection{Аксиома связи сложения и умножения}

\[
\forall a, b, c \in \mathbb{R} : (a + b) * c = a * c + b * c
\]

\subsection{Отношение порядка}

На $\mathbb{R}$ существует отношение $\ge$, подчиняющееся следущим аксиомам:

\begin{enumerate}
	\item \textbf{Рефлексивность}: $a \ge a$
	\item \textbf{Антисимметричность}: $a \ge b \land b \ge a \Rightarrow a = b$
	\item \textbf{Транзитивность}: $a \ge b \land b \ge c \Rightarrow a \ge c$
	\item \textbf{Аксиома, определяющая что порядок --- линейный}: \\
	$\forall a, b \in \mathbb{R} : a \ge b \lor b \ge a$
\end{enumerate}

{\large Замечание:}

\[ a \ge b \Rightarrow b \ge a \]
\[ a \ge b \land a \ne b \Rightarrow a > b \]

\subsection{Аксиома связи сложения и порядка}

\[ a \ge b \Rightarrow a + c \ge b + c \]

\subsection{Аксиома связи умножения и порядка}

\[ a \ge 0 \land b \ge 0 \Rightarrow a * b \ge 0 \]

\subsection{Аксиома полноты}

Пусть $U, W$ --- такие множества, что $\forall x \in U, y \in W : x \le y$

Тогда $\exists c \in \mathbb{R} : x \le c \le y$

\pagebreak
\section{Альтернативные зависимости}

\subsection{Полярная система координат}

$r$ --- расстояние от начала координат

$\phi$ --- угол от оси $Ox$ против часовой стрелки

\subsubsection*{Примеры:}

\begin{enumerate}
	\item Прямая $y = 1$
	\[ r = \frac{1}{sin \phi} \]
	\item Окружность $(x - 1)^2 + y^2 = 1$
	\[ r = 2 cos \phi \]
\end{enumerate}

\subsection{Параметрическая}

\begin{align*}
	y &= \phi(t) \\
	x &= \psi(t)
\end{align*}

\subsubsection*{Примеры:}

\begin{enumerate}
	\item Окружность $x^2 + y^2 = 1$
	\[
	\begin{cases}
		x = cos(t) \\
		y = sin(t)
	\end{cases}
	\]
	\item Отрезок $x + y = 2, x \ge 0, y \ge 0$
	\[
	\begin{cases}
		x = 2 cos^2 t \\
		y = 2 sin^2 t
	\end{cases}
	\]
\end{enumerate}

\section{Натуральные числа}

\textbf{Индуктивное множество} --- множество, обладающие тем \\
свойством, что наряду с элементом $a$, ему принадлежит элемент $a + 1$.

\subsection{Определения}

Множество \textbf{натуральных чисел} --- минимальное индуктивное \\
множество, содержащее единицу. То есть пересечение всех индуктивных множеств, \\
содержащих единицу.

\[
\mathbb{N} = \bigcap M : 1 \in M \land (x \in M \Rightarrow (x + 1) \in M)
\]

Множество $A$ называется \textbf{ограниченным сверху}, если $\exists b \in \mathbb{R} : \forall a \in A, a \le b$.

Тогда $b$ --- \textbf{мажората} $A$.

Множество $A$ называется \textbf{ограниченным снизу}, если $\exists c \in \mathbb{R} : \forall a \in A, a \ge c$.

Тогда $c$ --- \textbf{минората} $A$.

$d$ --- \textbf{максималный элемент} множества $A$, если $d \in A \land \forall a \in A : a \le d$.

$e$ --- \textbf{минимальный элемент} множества $A$, если $e \in A \land \forall a \in A : a \ge e$.

\textbf{Точная верхняя грань (супремум)} множества $A$ --- наименьшая можоранта $A$

$\sup A$ --- супремум $A$

Любое ограниченное сверху множество имеет супремум.

\underline{Доказательство:}

\begin{gather*}
	B = \{b \in \mathbb{R}\ |\ \forall a \in A, b \ge a\} \\
	\forall a \in A, b \in B : a \le b \stackrel{A16}{\Rightarrow} \\
	\stackrel{A16}{\Rightarrow} \exists c \in \mathbb{R} :
	\forall a \in A, b \in B \Rightarrow a \le c \le b \Rightarrow \\
	\Rightarrow c \stackrel{\triangle}{=} \sup A
\end{gather*}

\textbf{Точная нижняя грань (инфимум)} множества $A$ --- наибольшая минората $A$

$\inf A$ --- инфимум $A$

Любое ограниченное снизу множество имеет инфимум.

\pagebreak
\underline{Теорема:}
\[
	\forall x \in \mathbb{R}, x > 0 \exists n \in \mathbb{N} : n - 1 < x \le n
\]

\underline{Принцип Архимеда}
\[
\forall h > 0 \forall x \in \mathbb{R}, x > 0 \exists n \in \mathbb{N} : (n - 1)h < x \le nh
\]

\section{Пределы, последовательности}

\subsection{Последовательности}

\textbf{Последовательность} --- отображение $\mathbb{N}$ в множество любой \\
природы.

\textbf{Числовая последовательность} --- отображение $\mathbb{N}$ в $\mathbb{R}$.
\[ \{ a_n \}, a_n \in \mathbb{R}, n \in \mathbb{N} \]

Последовательность называется \textbf{ограниченной}, если множество ее значений ограниченно.

\subsection{Покрытие}

\begin{gather*}
	n \rightarrow I_n (\text{$I_n$ --- отрезок действительной оси}) \\
	\{ I_n \} : \forall n \in \mathbb{N}, I_{n + 1} \subset I_n \Rightarrow \\
	\Rightarrow \{ I_n \}
\end{gather*}

\underline{Теорема:}

Пусть $\{ U_n \}, n \in \mathbb{N}$ --- система вложенных отрезков.

Тогда $\exists c \in \mathbb{R}, \forall n \in \mathbb{N} : c \in U_n$.

Кроме того, если $U_n = [a_n, b_n]$ и \\
$\forall \epsilon > 0, \exists n \in \mathbb{N} : b_n - a_n < \epsilon$, то $\exists ! c$

\textbf{Покрытие} $B$ --- система множеств $A_1, A_2, \dots A_n$, если \\
$\forall x \in B, \exists n : x \in A_n$ 

Или

\[
B \subseteq \bigcup \limits_{i=1}^{n} A_i
\]

\underline{Теорема:}

Из любого покрытия отрезка интервалами можно выделить конечное подпокрытие

\subsection{Определение предела}
\begin{gather*}
	\{x_n\}_{n = 1, 2 \dots} \\
	\lim_{n \to \infty} x_n = a \\
	\text{if } \forall \epsilon > 0 \exists N(\epsilon) : \\
	|x_n - a| < \epsilon, n > N(\epsilon)
\end{gather*}

\underline{Теорема:}

У любой последовательности может быть не более 1 предела

\underline{Доказательство:}

\begin{gather*}
	\text{let } \exists lim_{n \to \infty} a_n = A_1 \land \\
	\land \exists lim_{n \to \infty} a_n = A_2 \land A_1 < A_2 \\
	\text{let } \epsilon = \frac{A_2 - A_1}{3} \\
	\exists N_1 : \forall n > N_1, |a_N - A_1| < \epsilon \Leftrightarrow
	A_1 - \epsilon < a_n < A_1 + \epsilon \\
	\exists N_2 : \forall n > N_2, |a_N - A_2| < \epsilon \Leftrightarrow
	A_2 - \epsilon < a_n < A_2 + \epsilon \\
	\text{let } N = \max\{N_1, N_2\} : \forall n > N, \\
	a_n < A_1 + \epsilon = A_1 + \frac{A_2 - A_1}{3} = \frac{2A_1 - A_2}{3} < \frac{A_1 + 2A_2}{3} = A_2 - \epsilon < a_n \\
	a_n < a_n
\end{gather*}

\underline{Замечание:}

Ограниченная последовательность может не иметь предела.

\subsubsection*{Пример:}

Доказать:
\[
	\lim_{n \to \infty} \frac{2x + 3}{n + 1} = 2
\]

Доказательство:
\begin{gather*}
	|\frac{2n + 3}{n + 1} - 2| = |\frac{2n + 3 - 2n - 2}{n + 1}| = |\frac{1}{n + 1}| < \epsilon \\
	n + 1 > \frac{1}{\epsilon} \\
	n > \frac{1}{\epsilon} - 1 \\
	N(\epsilon) = \frac{1}{\epsilon} - 1
\end{gather*}

\subsubsection{Бесконечные пределы}

\[
(\lim_{n \to \infty} a_n) = \infty \stackrel{def}{\Leftrightarrow} (\forall E > 0 \exists N(E) : \forall n > N : |a_n| > E)
\]
\[
(\lim_{n \to \infty} a_n) = -\infty \stackrel{def}{\Leftrightarrow} (\forall E > 0 \exists N(E) : \forall n > N : |a_n| < -E)
\]

\textbf{Бесконечно большая} последовательность --- последовательность, предел которой --- $\pm \infty$.

\textbf{Сходящаяся} последовательность --- последовательность, предел \\
которой --- конечное число.

\textbf{Бесконечно малая} последовательность --- последовательность, \\
предел которой равен нулю.

\subsubsection{Критерий Коши}

\subsection{Геометрическая прогрессия}

\begin{gather*}
	x_n = a \cdot q^{n-1} \\
	\lim_{n \to \inf} = \frac{a}{1 - q}
\end{gather*}

\subsubsection*{Примеры:}

\begin{gather*}
	\lim_{n \to \infty} \sum_{k=1}^{n-1} \frac{k - 1}{n^2} = \lim_{n \to \infty} \frac{1}{n^2} \sum_{k=1}^{n-1} k - 1 = \\
	= \lim_{n \to \infty} \frac{1}{n^2} \frac{(1 + (n - 1))(n - 1)}{2} = \lim_{n \to \infty} \frac{n^2 - n}{2n^2} = \\
	\frac{1}{2} + \lim_{n \to \infty} = \frac{1}{2} + 0 = \frac{1}{2}
\end{gather*}

\begin{gather*}
	\lim_{n \to \infty} \frac{(n + 1)(n + 2)(n + 3)}{n^3} = \\
	\lim_{n \to \infty} \frac{n(1 + \frac{1}{n}) \cdot n(1 + \frac{2}{n}) \cdot n(1 + \frac{3}{n})}{n^3} = \\
	\lim_{n \to \infty} (1 + \frac{1}{n})(1 + \frac{2}{n})(1 + \frac{3}{n}) = 1
\end{gather*}

\subsection{Фундаментальная последовательность. \\
Критерий Коши}

Последовательность $\{ a_n \}$ называется \textbf{фундаментальной}, если \[
\forall \epsilon > 0 \exists N(\epsilon) : \forall m > n > N : |a_m - a_n| < \epsilon
\]

\subsubsection{Критерий коши}

Последовательность является фундаментальной $\Leftrightarrow$ \\
$\Leftrightarrow$ последовательность является сходящейся

\underline{Доказательство $1 \Rightarrow 2$:}
\begin{gather*}
	\exists \lim_{n \to \infty} a_n = A, A \in \mathbb{R} \\
	\forall \epsilon > 0 \exists N(\epsilon) \forall n > N, n \in \mathbb{N} :
	|a_n - A| < \frac{\epsilon}{2} \\
	\forall \epsilon > 0 \exists N(\epsilon) \forall m > N, m \in \mathbb{N} :
	|a_m - A| < \frac{\epsilon}{2} \\
	|a_n - a_m| = |a_n - A - (a_m - A)| \le |a_n - A| + |a_m - A| < \frac{\epsilon}{2} + \frac{\epsilon}{2} = \epsilon
\end{gather*}

\underline{Доказательство $1 \Leftarrow 2$:}
\begin{gather*}
	\forall \epsilon > 0 \exists N(\epsilon) : \forall m > n > N : |a_n - a_m| < \frac{\epsilon}{3} \\
	\text{let } x_n =\inf_{m \ge n} a_m \\
	\text{let } y_n =\sup_{m \ge n} a_m \\
	x_n \le x_{n+1} \le y_{n+1} \le y_n \\
	\{ [x_n, y_n] \} \text{ --- система вложенных отрезков} \Rightarrow \exists c : \forall n \in \mathbb{N}, c \in [x_n, y_n] \\
	m > n \\
	a_n - \frac{\epsilon}{3} < a_m < a_n + \frac{\epsilon}{3} \\
	a_n - \frac{\epsilon}{3} \le x_m \le c \le y_m \le a_n + \frac{\epsilon}{3} \\
	|a_n - c| \le \frac{\epsilon}{3} \\
	|a_m - a_n| < \frac{\epsilon}{3} \\
	|x_m - a_n| \le \frac{\epsilon}{3} \\
	|y_m - a_n| \le \frac{\epsilon}{3} \\
	|a_m - c| \le |a_m - a_n + a_n - c| \le |a_m - a_n| + |a_n - c| \le
	\frac{\epsilon}{3} + \frac{\epsilon}{3} = \frac{2\epsilon}{3} < \epsilon
\end{gather*}

\subsection{Алгебраические свойства пределов}

\begin{gather*}
	\lim_{n \to \infty} a_n = A \\
	\lim_{n \to \infty} b_n = B \\
\end{gather*}

\subsubsection{Сложение}

\begin{gather*}
	(\lim_{n \to \infty} a_n = A) \Leftrightarrow (\forall \epsilon > 0 \exists N_1(\epsilon) :
	\forall n > N_1, n \in \mathbb{N} : |a_n - A| < \frac{\epsilon}{2}) \\
	(\lim_{n \to \infty} b_n = B) \Leftrightarrow (\forall \epsilon > 0 \exists N_2(\epsilon) :
	\forall n > N_2, n \in \mathbb{N} : |b_n - B| < \frac{\epsilon}{2}) \\
	\text{let } N = \max\{N_1, N_2\} \\
	\forall n > N : |a_n + b_n - (A + B)| \le |a_n - A| + |b_n - B| \le
	\frac{\epsilon}{2} + \frac{\epsilon}{2} = \epsilon \Leftrightarrow \\
	\Leftrightarrow \lim_{n \to \infty} (a_n + b_n) = A + B
\end{gather*}

\subsubsection{Умножение}

\begin{gather*}
	a_n \cdot b_n = (a_n - A + A)(b_n - B + B) = \\
	= (a_n - A)(b_n - B) + A(b_n - B) + B(a_n - A) + AB \\
	\forall \epsilon > 0 \exists N_1(\epsilon) : \forall n > N_1 : |a_n - A| < \epsilon \\
	\forall \epsilon > 0 \exists N_2(\epsilon) : \forall n > N_2 : |b_n - B| < \epsilon \\
	\forall \epsilon > 0 \exists N_3(\epsilon) : \forall n > N_3 : |b_n - B| < \frac{\epsilon}{3|A|} \\
	\forall \epsilon > 0 \exists N_4(\epsilon) : \forall n > N_4 : |a_n - A| < \frac{\epsilon}{3|B|} \\
	\text{let } N = \max\{N_1, N_2, N_3, N_4\} \\
	\forall n > N : |a_n b_n - AB| \le |a_n - A||b_n - B| + |A||b_n - B| + |B||a_n - A| < \\
	< \epsilon^2 + |A| \frac{\epsilon}{3|A|} + |B| \frac{\epsilon}{3|B|} < 3 \cdot \frac{\epsilon}{3} = \epsilon \\
	(\epsilon^2 < \frac{\epsilon}{3} \Leftarrow \epsilon < \frac{1}{3})
\end{gather*}

\subsubsection{Деление}

\begin{gather*}
	\frac{a_n}{b_n} - \frac{A}{B} = \frac{a_nB - b_nA}{b_nB} = \\
	= \frac{(a_n - A)B - (b_n - B)A}{b_nB} = \frac{a_n - A}{b_n} - \frac{b_n - B}{b_n} \cdot \frac{A}{B} \\
	\exists N_1 : \forall n > N_1 : |b_n| > \frac{B}{2} \\
	\forall \epsilon > 0 \exists N_2 : \forall n > N_2 : |a_n - A| < \frac{\epsilon|B|}{4} \\
	\forall \epsilon > 0 \exists N_3 : \forall n > N_3 : |b_n - B| < \frac{\epsilon|B|^2}{4|A|} \\
	\text{let } N = \max\{N_1, N_2, N_3\} \\
	\forall n > N : |\frac{a_n}{b_n} - \frac{A}{B}| \le \frac{|a_n - A|}{|b_n|} +
	\frac{|b_n - B|}{|b_n|} \cdot \frac{|A|}{|B|} < \\
	< \frac{2\epsilon|B|}{4|B|} + \frac{2\epsilon|B|^2}{4|A||B|} \cdot \frac{|A|}{|B|} =
	\frac{\epsilon}{2} + \frac{\epsilon}{2} = \epsilon
\end{gather*}

\subsection{Принцип двух полицейских}

\begin{gather*}
	\begin{cases}
		\exists \lim_{n \to \infty} a_n = A \\
		\exists \lim_{n \to \infty} b_n = A \\
		\forall n \in \mathbb{N} : a_n \le c_n \le b_n
	\end{cases} \Rightarrow \lim_{c \to \infty} c_n = A
\end{gather*}

\subsection{Возрастающие и убывающие посл.}

Последовательность $\{ a_n \}$ называется \textbf{возрастающей}, если $\forall n \in \mathbb{N} : a_n < a_{n + 1}$

Последовательность $\{ a_n \}$ называется \textbf{неубывающей}, если $\forall n \in \mathbb{N} : a_n \le a_{n + 1}$

Последовательность $\{ a_n \}$ называется \textbf{убывающей}, если $\forall n \in \mathbb{N} : a_n > a_{n + 1}$

Последовательность $\{ a_n \}$ называется \textbf{невозрастающей}, если $\forall n \in \mathbb{N} : a_n \ge a_{n + 1}$

\subsection{Теорема Вейерштрасса}

Монотонная ограниченная последовательность имеет конечный \\
предел.

\underline{Доказательство:}
\begin{gather*}
	\text{let } a_n : \forall n \in \mathbb{N} : a_{n + 1} \ge a_n \\
	\exists A : \forall n \in \mathbb{N} : a_n < A \\
	\lim_{n \to \infty} a_n = a \\
	\exists c = \sup_{n \in \mathbb{N}} a_n : c - \epsilon < a_n \le c \\
	\forall m > n : c - \epsilon < a_n \le a_m \le c 
	\stackrel{\triangle}{\Rightarrow} \lim_{n \to \infty} a_n = c
\end{gather*}

\subsubsection{Условие существования предела.}

Для того, чтобы монотонная последовательность имела предел, \\
необходимо и достаточно, чтобы она была ограничена.

\section{Пределы функции}

\[
\lim_{n \to a} f(x) = A \text{ --- const} \lor \pm \infty \lor \lnot \exists
\]

\subsection*{Примеры:}
\begin{gather*}
	\lim_{x \to \frac{\pi}{2}} sin x = 1 \\
	\lim_{x \to \infty} sin x = \lnot \exists \\
	\lim_{x \to \infty} \frac{x^2 + 5x}{x + 1} \stackrel{: x^2}{=}
	\lim_{x \to \infty} \frac{1 + \frac{5}{x}}{\frac{1}{x} + \frac{1}{x^2}} =
	\lim_{x \to \infty} \frac{1 + 0}{0 + 0} = \infty \\
	\lim_{x \to \infty} \frac{2x^2 - 3x - 4}{\sqrt{x^4 + 1}} \stackrel{: x^2}{=}
	\lim_{x \to \infty} \frac{2 - \frac{3}{x} - \frac{4}{x^2}}{\sqrt{1 + \frac{1}{x^4}}} =
	\lim_{x \to \infty} \frac{2 + 0 + 0}{\sqrt{1 + 0}} = 2 \\
	\lim_{x \to \infty} \frac{x\sqrt{x} + x^2}{x\sqrt{x} + x} \stackrel{: x^2}{=}
	\lim_{x \to \infty} \frac{\frac{1}{\sqrt{x}} + 1}{\frac{1}{\sqrt{x}} + \frac{1}{x}} =
	\frac{1}{0} = \infty
\end{gather*}

\subsection{Правило старшей степени}
\[
	\lim_{x \to \infty} \frac{P_m(x)}{P_n(x)} = \begin{cases}
		0, m < n \\
		\infty, m > n \\
		A, m = n
	\end{cases}
\]

\subsection{Неопределенность $\frac{0}{0}$}
\[
	\lim_{x \to a} \frac{P(x)}{Q(x)} = \left[\frac{0}{0}\right] =
	\frac{(x - a)P'(x)}{(x-a)Q'(x)} = \frac{P'(x)}{Q'(x)} = A
\]

\subsection{Неопределенность $\infty - \infty$}
\[
	\lim_{x \to a} [P(x) - Q(x)] = [\infty - \infty] \rightarrow \lim_{x \to a} \left[\frac{0}{0}\right] \dots
\]

\subsection{Иррациональность}
\begin{align*}
	\lim_{x \to 1} \frac{\sqrt{x} - 1}{x - 1} &= \lim_{x \to 1} \frac{x - 1}{(x - 1)(\sqrt{x} + 1)} \\
	&= \lim_{x \to 1} \frac{\sqrt{x} - 1}{(\sqrt{x} - 1)(\sqrt{x} + 1)} \\
	&= \{ t = \sqrt{x} \} = \lim_{t \to 1} \frac{t - 1}{t^2 - 1}
\end{align*}

\subsection{Первый замечательный предел}
\[
\lim_{x \to 0} \frac{\sin{x}}{x} = 1
\]

\subsubsection*{Примеры:}

\[
\lim_{x \to 0} \frac{1 - \cos{x}}{x^2} = \left[\frac{0}{0}\right] =
\lim_{x \to 0} \frac{1 - \cos^2{x}}{x^2(1 + \cos{x})} =
\lim_{x \to 0} \frac{\sin^2{x}}{x^2(1 + \cos{x})} = \frac{1}{2}
\]

\end{document}