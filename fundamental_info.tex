\documentclass{article}
\usepackage[14pt]{extsizes}
\usepackage[a4paper, margin=2.5cm]{geometry}
\usepackage{indentfirst}
\usepackage[T1, T2A]{fontenc}
\usepackage[russian,english]{babel}
\usepackage{amsmath}
\usepackage{amssymb}

\begin{document}

\title{Фундаментальная информатика}
\author{Илья Ковалев}
\date{2024 год}
\maketitle

\section{Информатика}

Информатика – наука, связанная с автоматизацией. Есть английское и французкое определения.

\subsection{Информация и сообщение}

Информация и сообщение --- основные неопределяемые понятия.

Информация --- то, что передается в сообщении.

Сообщение --- то, что передает информацию.

Информация может существовать только в материальной форме.

У каждого предмета есть информация, так же как масса и энергия.

Интерпретация сообщения --- понимание, осмысливание.

\subsection{Правила интерпретации}

\begin{itemize}
	\item{Каждое сообщение $n$ содержит информацию $i$}
	\item{Существует инъекция, преобразующая разные сообщения в разные сведения. Функция интерпретации $\phi: N \rightarrow I$. $\phi$ --- конъюктивное отображение, сопостовляющее каждому правильному $n$ его $i$.}
	\item{Если множество $N$ конечно, то $N$ --- конечный язык.\\
	Если $N$ бесконечно, существуют правила (рекурсивные).}
\end{itemize}

Например, красный, желтый и зеленый сигналы светофора --- это\\
конечное множество, следовательно это язык.

Перевод с иностранного языка на родной --- две инъекции:

$\pi: N_0 \rightarrow N_1$
$\phi: N_1 \rightarrow I$

\subsection{Сигналы}

Аналоговые среды/сигналы --- плавно меняющиеся,\\
непрерывные, континуальные.

Пример: часы со стрелками, звуки.

Дискретные --- не аналоговые.

Камера --- дискретное подобие аналогого глаза.

Микрофон и динамики --- аналоговые вход/выход, преобразующие аналоговый в дискретный (и наоборот) сигналы.

\subsection{Знаки и символы}

Среда звуковых сигналов --- воздух.\\
Его особенность --- недолговременность.

\textbf{Долговременные носители (письменные)} --- бумага, магнитная лента, и.т.д.

\textbf{Запись} --- процесс сохранения на письменные носители.

\textbf{Чтение} --- процесс воспроизведения .

\textbf{Знаки} --- элементы письменности.

\textbf{Атомарные знаки (буквы)} --- элемент алфавита,\\
различимый графическим способом литер.

\textbf{Алфавит} --- упорядоченное конечное непустое множество\\
допустимых знаков в сообщении.

Пример атомарных знаков: русские буквы (33), цифры (0-9), знаки + - / *.

Примеры алфовитов: русский алфавит, латиница, брайль.

\textbf{Составные знаки (слова)} --- конечная последовательность знаков (атомарных или составных). Может быть пустым.

\textbf{Письменное сообщение} --- набор атомарных знаков,\\
или набор составных знаков, или единственный составной знак.

Для отделения составных знаков нужен "пробел".

Для различения k уровней знаков нужен k - 1 разделитель.

Разделители в разговорном языке --- пробелы, точки, \\
восклицательные знаки, абзацы, страницы, и.т.д.

Разные знаки могут обозначать одно и то же значение, и наоборот.\\
Например, $\cdot$ и $\times$ обозначают умножение.

\subsection{Кодирование}

$A$ --- алфавит.

Кодирование --- отображение алфовитов $C: A \rightarrow A'$ (сюрьекция)

\subsection{Алгоритмы}

\subsubsection{Машина Тьюринга}

--- ограниченная с одной стороны бесконечная лента,\\
в каждой ячейке записан символ из алфавита $A$.

В определенный момент времени головка находится в одной из ячеек\\
ленты и имеет одно дискретное состояние из набора возможных.

Формальное определение:

\textbf{Машина Тьюринга} --- упорядоченная четверка объектов\\
$T = (A, Q, P, q_0)$, где $T$ --- элемент МТ.

\begin{itemize}
	\item $q_0$ --- начальное состояние головки
	\item $q_1$ --- новое состояние головки
	\item $a_0 \in A$ --- начальное значение клетки
	\item $a_1 \in A$ --- новое значение клетки
	\item $v$ --- передвижение головки
\end{itemize}

\end{document}
